%%
%% Automatically generated file from DocOnce source
%% (https://github.com/doconce/doconce/)
%% doconce format latex PDE.do.txt --print_latex_style=trac --latex_admon=paragraph
%%
% #ifdef PTEX2TEX_EXPLANATION
%%
%% The file follows the ptex2tex extended LaTeX format, see
%% ptex2tex: https://code.google.com/p/ptex2tex/
%%
%% Run
%%      ptex2tex myfile
%% or
%%      doconce ptex2tex myfile
%%
%% to turn myfile.p.tex into an ordinary LaTeX file myfile.tex.
%% (The ptex2tex program: https://code.google.com/p/ptex2tex)
%% Many preprocess options can be added to ptex2tex or doconce ptex2tex
%%
%%      ptex2tex -DMINTED myfile
%%      doconce ptex2tex myfile envir=minted
%%
%% ptex2tex will typeset code environments according to a global or local
%% .ptex2tex.cfg configure file. doconce ptex2tex will typeset code
%% according to options on the command line (just type doconce ptex2tex to
%% see examples). If doconce ptex2tex has envir=minted, it enables the
%% minted style without needing -DMINTED.
% #endif

% #define PREAMBLE

% #ifdef PREAMBLE
%-------------------- begin preamble ----------------------

\documentclass[%
oneside,                 % oneside: electronic viewing, twoside: printing
final,                   % draft: marks overfull hboxes, figures with paths
10pt]{article}

\listfiles               %  print all files needed to compile this document

\usepackage{relsize,makeidx,color,setspace,amsmath,amsfonts,amssymb}
\usepackage[table]{xcolor}
\usepackage{bm,ltablex,microtype}

\usepackage[pdftex]{graphicx}

\usepackage[T1]{fontenc}
%\usepackage[latin1]{inputenc}
\usepackage{ucs}
\usepackage[utf8x]{inputenc}

\usepackage{lmodern}         % Latin Modern fonts derived from Computer Modern

% Hyperlinks in PDF:
\definecolor{linkcolor}{rgb}{0,0,0.4}
\usepackage{hyperref}
\hypersetup{
    breaklinks=true,
    colorlinks=true,
    linkcolor=linkcolor,
    urlcolor=linkcolor,
    citecolor=black,
    filecolor=black,
    %filecolor=blue,
    pdfmenubar=true,
    pdftoolbar=true,
    bookmarksdepth=3   % Uncomment (and tweak) for PDF bookmarks with more levels than the TOC
    }
%\hyperbaseurl{}   % hyperlinks are relative to this root

\setcounter{tocdepth}{2}  % levels in table of contents

% prevent orhpans and widows
\clubpenalty = 10000
\widowpenalty = 10000

% --- end of standard preamble for documents ---


% insert custom LaTeX commands...

\raggedbottom
\makeindex
\usepackage[totoc]{idxlayout}   % for index in the toc
\usepackage[nottoc]{tocbibind}  % for references/bibliography in the toc

%-------------------- end preamble ----------------------

\begin{document}

% matching end for #ifdef PREAMBLE
% #endif

\newcommand{\exercisesection}[1]{\subsection*{#1}}


% ------------------- main content ----------------------



% ----------------- title -------------------------

\thispagestyle{empty}

\begin{center}
{\LARGE\bf
\begin{spacing}{1.25}
Project 1, Partial differential equations with Neural Networks
\end{spacing}
}
\end{center}

% ----------------- author(s) -------------------------

\begin{center}
{\bf \href{{https://www.uio.no/studier/emner/matnat/fys/FYS5429/index-eng.html}}{FYS5429/9429}, Advanced machine learning and data analysis for the physical sciences, University of Oslo, Norway${}^{}$} \\ [0mm]
\end{center}

\begin{center}
% List of all institutions:
\end{center}
    
% ----------------- end author(s) -------------------------

% --- begin date ---
\begin{center}
Spring semester 2025, deadline March 21
\end{center}
% --- end date ---

\vspace{1cm}


\subsection{Solving partial differential equations with neural networks}

This variant of project 1 is tailored to those of you who are
interested in studying differential equations and may have followed
popular courses on these methods. \textbf{It can also be seen as a stepping
stone towards studies of PINNs, laying thereby  the basis for
project 2}.

For this variant of project 1, we will assume that you have some
background in the solution of partial differential equations using
finite difference schemes. If you are not familiar with these methods, we can give you an introduction.

We will study the solution of the diffusion
equation in one dimension using a standard explicit scheme and neural
networks to solve the same equations. Feel free to add more advanced finite difference or finite element methods.

For the explicit scheme, you can study for example chapter 10 of the lecture notes in \href{{https://github.com/CompPhysics/ComputationalPhysics/blob/master/doc/Lectures/lectures2015.pdf}}{Computational Physics, FYS3150/4150} or alternative sources from courses like \href{{https://www.uio.no/studier/emner/matnat/math/MAT-MEK4270/index.html}}{MAT-MEK4270}. For the solution of ordinary and partial differential equations using neural networks, the lectures by of week 43 at for example \href{{https://compphysics.github.io/MachineLearning/doc/pub/week42/html/week43.html}}{\nolinkurl{https://compphysics.github.io/MachineLearning/doc/pub/week42/html/week43.html}} at this course are highly recommended.

For the machine learning part you can use your own codes
or the functionality of for example \textbf{Tensorflow/Keras}, \textbf{PyTorch} or
other libraries such as \href{{https://maziarraissi.github.io/PINNs/}}{Physics informed machine learning}.

\paragraph{Alternative differential equations.}
Note that you can replace the one-dimensional diffusion equation
discussed below with other sets of either ordinary differential
equations or partial differential equations.
A typical equation many of you may be interested in is for example the Navier-Stokes equation.

An alternative is a
stochastic diffusion equation, known as the Black-Scholes equation
(Nobel prize in economy, see
\href{{https://en.wikipedia.org/wiki/Black%E2%80%93Scholes_model}}{\nolinkurl{https://en.wikipedia.org/wiki/Black\%E2\%80\%93Scholes_model}}).

An interesting article on PINNs with the Black-Scholes equation could serve as a possible path for the second project, see \href{{https://arxiv.org/abs/2312.06711}}{\nolinkurl{https://arxiv.org/abs/2312.06711}}.

\paragraph{Part a), setting up the problem.}
The physical problem can be that of the temperature gradient in a rod of length $L=1$ at $x=0$ and $x=1$.
We are looking at a one-dimensional
problem

\begin{equation*}
 \frac{\partial^2 u(x,t)}{\partial x^2} =\frac{\partial u(x,t)}{\partial t}, t> 0, x\in [0,L]
\end{equation*}
or

\begin{equation*}
u_{xx} = u_t,
\end{equation*}
with initial conditions, i.e., the conditions at $t=0$,
\begin{equation*}
u(x,0)= \sin{(\pi x)} \hspace{0.5cm} 0 < x < L,
\end{equation*}
with $L=1$ the length of the $x$-region of interest. The 
boundary conditions are

\begin{equation*}
u(0,t)= 0 \hspace{0.5cm} t \ge 0,
\end{equation*}
and

\begin{equation*}
u(L,t)= 0 \hspace{0.5cm} t \ge 0.
\end{equation*}
The function $u(x,t)$  can be the temperature gradient of a  rod.
As time increases, the velocity approaches a linear variation with $x$. 

We will limit ourselves to the so-called explicit forward Euler algorithm with discretized versions of time given by a forward formula and a centered difference in space resulting in   
\begin{equation*} 
u_t\approx \frac{u(x,t+\Delta t)-u(x,t)}{\Delta t}=\frac{u(x_i,t_j+\Delta t)-u(x_i,t_j)}{\Delta t}
\end{equation*}
and

\begin{equation*}
u_{xx}\approx \frac{u(x+\Delta x,t)-2u(x,t)+u(x-\Delta x,t)}{\Delta x^2},
\end{equation*}
or

\begin{equation*}
u_{xx}\approx \frac{u(x_i+\Delta x,t_j)-2u(x_i,t_j)+u(x_i-\Delta x,t_j)}{\Delta x^2}.
\end{equation*}

Write down the algorithm and the equations you need to implement.
Find also the analytical solution to the problem. 

\paragraph{Part b).}
Implement the explicit scheme  algorithm and perform tests of the solution 
for $\Delta x=1/10$, $\Delta x=1/100$ using  $\Delta t$ as dictated by the stability limit of the explicit scheme. The stability criterion for the explicit scheme requires that $\Delta t/\Delta x^2 \leq 1/2$. 

Study the solutions at two time points $t_1$ and $t_2$ where $u(x,t_1)$ is smooth but still significantly curved
and $u(x,t_2)$ is almost linear, close to the stationary state.

\paragraph{Part c) Neural networks.}
Study now the lecture notes on solving ODEs and PDEs with neural
network and use either your own code from project 2 or the
functionality of tensorflow/keras to solve the same equation as in
part b).  Discuss your results and compare them with the standard
explicit scheme. Include also the analytical solution and compare with
that.

\paragraph{Part d) Neural network complexity.}
Here we study the stability of the results of the results as functions of the number of hidden nodes, layers and activation functions for the hidden layers.
Increase the number of hidden nodes and layers in order to see if this improves your results. Try also different activation functions for the hidden layers, such as the \textbf{tanh}, \textbf{ReLU}, and other activation functions. 
Discuss your results.

\paragraph{Part e).}
Finally, present a critical assessment of the methods you have studied
and discuss the potential for the solving differential equations with machine learning methods.

\subsection{Introduction to numerical projects}

Here follows a brief recipe and recommendation on how to write a report for each
project.

\begin{itemize}
  \item Give a short description of the nature of the problem and the eventual  numerical methods you have used.

  \item Describe the algorithm you have used and/or developed. Here you may find it convenient to use pseudocoding. In many cases you can describe the algorithm in the program itself.

  \item Include the source code of your program. Comment your program properly.

  \item If possible, try to find analytic solutions, or known limits in order to test your program when developing the code.

  \item Include your results either in figure form or in a table. Remember to        label your results. All tables and figures should have relevant captions        and labels on the axes.

  \item Try to evaluate the reliabilty and numerical stability/precision of your results. If possible, include a qualitative and/or quantitative discussion of the numerical stability, eventual loss of precision etc.

  \item Try to give an interpretation of you results in your answers to  the problems.

  \item Critique: if possible include your comments and reflections about the  exercise, whether you felt you learnt something, ideas for improvements and  other thoughts you've made when solving the exercise. We wish to keep this course at the interactive level and your comments can help us improve it.

  \item Try to establish a practice where you log your work at the  computerlab. You may find such a logbook very handy at later stages in your work, especially when you don't properly remember  what a previous test version  of your program did. Here you could also record  the time spent on solving the exercise, various algorithms you may have tested or other topics which you feel worthy of mentioning.
\end{itemize}

\noindent
\subsection{Format for electronic delivery of report and programs}

The preferred format for the report is a PDF file. You can also use DOC or postscript formats or as an ipython notebook file.  As programming language we prefer that you choose between C/C++, Fortran2008 or Python. The following prescription should be followed when preparing the report:

\begin{itemize}
  \item Send us an email in order  to hand in your projects with a link to your GitHub/Gitlab repository.

  \item In your GitHub/GitLab or similar repository, please include a folder which contains selected results. These can be in the form of output from your code for a selected set of runs and input parameters.
\end{itemize}

\noindent
Finally, 
we encourage you to collaborate. Optimal working groups consist of 
2-3 students. You can then hand in a common report. 


% ------------------- end of main content ---------------

% #ifdef PREAMBLE
\end{document}
% #endif

