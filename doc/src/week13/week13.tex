%%
%% Automatically generated file from DocOnce source
%% (https://github.com/doconce/doconce/)
%% doconce format latex week13.do.txt --minted_latex_style=trac --latex_admon=paragraph --no_mako
%%


%-------------------- begin preamble ----------------------

\documentclass[%
oneside,                 % oneside: electronic viewing, twoside: printing
final,                   % draft: marks overfull hboxes, figures with paths
10pt]{article}

\listfiles               %  print all files needed to compile this document

\usepackage{relsize,makeidx,color,setspace,amsmath,amsfonts,amssymb}
\usepackage[table]{xcolor}
\usepackage{bm,ltablex,microtype}

\usepackage[pdftex]{graphicx}

% Movies are handled by the href package
\newenvironment{doconce:movie}{}{}
\newcounter{doconce:movie:counter}


\usepackage{fancyvrb} % packages needed for verbatim environments
\usepackage{minted}
\usemintedstyle{default}

\usepackage[T1]{fontenc}
%\usepackage[latin1]{inputenc}
\usepackage{ucs}
\usepackage[utf8x]{inputenc}

\usepackage{lmodern}         % Latin Modern fonts derived from Computer Modern

% Hyperlinks in PDF:
\definecolor{linkcolor}{rgb}{0,0,0.4}
\usepackage{hyperref}
\hypersetup{
    breaklinks=true,
    colorlinks=true,
    linkcolor=linkcolor,
    urlcolor=linkcolor,
    citecolor=black,
    filecolor=black,
    %filecolor=blue,
    pdfmenubar=true,
    pdftoolbar=true,
    bookmarksdepth=3   % Uncomment (and tweak) for PDF bookmarks with more levels than the TOC
    }
%\hyperbaseurl{}   % hyperlinks are relative to this root

\setcounter{tocdepth}{2}  % levels in table of contents

% --- fancyhdr package for fancy headers ---
\usepackage{fancyhdr}
\fancyhf{} % sets both header and footer to nothing
\renewcommand{\headrulewidth}{0pt}
\fancyfoot[LE,RO]{\thepage}
% Ensure copyright on titlepage (article style) and chapter pages (book style)
\fancypagestyle{plain}{
  \fancyhf{}
  \fancyfoot[C]{{\footnotesize \copyright\ 1999-2023, Morten Hjorth-Jensen. Released under CC Attribution-NonCommercial 4.0 license}}
%  \renewcommand{\footrulewidth}{0mm}
  \renewcommand{\headrulewidth}{0mm}
}
% Ensure copyright on titlepages with \thispagestyle{empty}
\fancypagestyle{empty}{
  \fancyhf{}
  \fancyfoot[C]{{\footnotesize \copyright\ 1999-2023, Morten Hjorth-Jensen. Released under CC Attribution-NonCommercial 4.0 license}}
  \renewcommand{\footrulewidth}{0mm}
  \renewcommand{\headrulewidth}{0mm}
}

\pagestyle{fancy}


\usepackage[framemethod=TikZ]{mdframed}

% --- begin definitions of admonition environments ---

% --- end of definitions of admonition environments ---

% prevent orhpans and widows
\clubpenalty = 10000
\widowpenalty = 10000

% --- end of standard preamble for documents ---


% insert custom LaTeX commands...

\raggedbottom
\makeindex
\usepackage[totoc]{idxlayout}   % for index in the toc
\usepackage[nottoc]{tocbibind}  % for references/bibliography in the toc

%-------------------- end preamble ----------------------

\begin{document}

% matching end for #ifdef PREAMBLE

\newcommand{\exercisesection}[1]{\subsection*{#1}}


% ------------------- main content ----------------------



% ----------------- title -------------------------

\thispagestyle{empty}

\begin{center}
{\LARGE\bf
\begin{spacing}{1.25}
April 24-28: Advanced machine learning and data analysis for the physical sciences
\end{spacing}
}
\end{center}

% ----------------- author(s) -------------------------

\begin{center}
{\bf Morten Hjorth-Jensen${}^{1, 2}$} \\ [0mm]
\end{center}

\begin{center}
% List of all institutions:
\centerline{{\small ${}^1$Department of Physics and Center for Computing in Science Education, University of Oslo, Norway}}
\centerline{{\small ${}^2$Department of Physics and Astronomy and Facility for Rare Isotope Beams, Michigan State University, East Lansing, Michigan, USA}}
\end{center}
    
% ----------------- end author(s) -------------------------

% --- begin date ---
\begin{center}
April 24-28, 2023
\end{center}
% --- end date ---

\vspace{1cm}


\subsection*{Plans for the week April 24-28}


% --- begin paragraph admon ---
\paragraph{Deep generative models and Restricted Boltzmann machines.}
\begin{enumerate}
\item Restricted Boltzmann machines

\item Generative Adversarial Networks (GANs)

\item Reading recommendation: Goodfellow et al chapter 20.10-20-14

\item \href{{https://youtu.be/}}{Video of lecture}

\item \href{{https://github.com/CompPhysics/AdvancedMachineLearning/blob/main/doc/HandwrittenNotes/NotesApr262023.pdf}}{Whiteboard notes}
\end{enumerate}

\noindent
% --- end paragraph admon ---



% !split
\section*{Boltzmann Machines}

Why use a generative model rather than the more well known discriminative deep neural networks (DNN)? 

\begin{itemize}
\item Discriminitave methods have several limitations: They are mainly supervised learning methods, thus requiring labeled data. And there are tasks they cannot accomplish, like drawing new examples from an unknown probability distribution.

\item A generative model can learn to represent and sample from a probability distribution. The core idea is to learn a parametric model of the probability distribution from which the training data was drawn. As an example
\begin{enumerate}

 \item A model for images could learn to draw new examples of cats and dogs, given a training dataset of images of cats and dogs.

 \item Generate a sample of an ordered or disordered Ising model phase, having been given samples of such phases.

 \item Model the trial function for Monte Carlo calculations

\end{enumerate}

\noindent
\item Both use gradient-descent based learning procedures for minimizing cost functions

\item Energy based models don't use backpropagation and automatic differentiation for computing gradients, instead turning to Markov Chain Monte Carlo methods.

\item DNNs often have several hidden layers. A restricted Boltzmann machine has only one hidden layer, however several RBMs can be stacked to make up Deep Belief Networks, of which they constitute the building blocks.
\end{itemize}

\noindent
History: The RBM was developed by amongst others Geoffrey Hinton, called by some the "Godfather of Deep Learning", working with the University of Toronto and Google.


% --- begin paragraph admon ---
\paragraph{}
A BM is what we would call an undirected probabilistic graphical model
with stochastic continuous or discrete units.
% --- end paragraph admon ---



% --- begin paragraph admon ---
\paragraph{}
It is interpreted as a stochastic recurrent neural network where the
state of each unit(neurons/nodes) depends on the units it is connected
to. The weights in the network represent thus the strength of the
interaction between various units/nodes.
% --- end paragraph admon ---



% --- begin paragraph admon ---
\paragraph{}
It turns into a Hopfield network if we choose deterministic rather
than stochastic units. In contrast to a Hopfield network, a BM is a
so-called generative model. It allows us to generate new samples from
the learned distribution.
% --- end paragraph admon ---




% --- begin paragraph admon ---
\paragraph{}
A standard BM network is divided into a set of observable and visible units $\hat{x}$ and a set of unknown hidden units/nodes $\hat{h}$.
% --- end paragraph admon ---




% --- begin paragraph admon ---
\paragraph{}
Additionally there can be bias nodes for the hidden and visible layers. These biases are normally set to $1$.
% --- end paragraph admon ---




% --- begin paragraph admon ---
\paragraph{}
BMs are stackable, meaning they cwe can train a BM which serves as input to another BM. We can construct deep networks for learning complex PDFs. The layers can be trained one after another, a feature which makes them popular in deep learning
% --- end paragraph admon ---



However, they are often hard to train. This leads to the introduction of so-called restricted BMs, or RBMS.
Here we take away all lateral connections between nodes in the visible layer as well as connections between nodes in the hidden layer. 

\subsection*{The network}

\textbf{The network layers}:
\begin{enumerate}
 \item A function $\mathbf{x}$ that represents the visible layer, a vector of $M$ elements (nodes). This layer represents both what the RBM might be given as training input, and what we want it to be able to reconstruct. This might for example be the pixels of an image, the spin values of the Ising model, or coefficients representing speech.

 \item The function $\mathbf{h}$ represents the hidden, or latent, layer. A vector of $N$ elements (nodes). Also called "feature detectors".
\end{enumerate}

\noindent
The goal of the hidden layer is to increase the model's expressive power. We encode complex interactions between visible variables by introducing additional, hidden variables that interact with visible degrees of freedom in a simple manner, yet still reproduce the complex correlations between visible degrees in the data once marginalized over (integrated out).

Examples of this trick being employed in physics: 
\begin{enumerate}
 \item The Hubbard-Stratonovich transformation

 \item The introduction of ghost fields in gauge theory

 \item Shadow wave functions in Quantum Monte Carlo simulations
\end{enumerate}

\noindent
\textbf{The network parameters, to be optimized/learned}:
\begin{enumerate}
 \item $\mathbf{a}$ represents the visible bias, a vector of same length as $\mathbf{x}$.

 \item $\mathbf{b}$ represents the hidden bias, a vector of same lenght as $\mathbf{h}$.

 \item $W$ represents the interaction weights, a matrix of size $M\times N$.
\end{enumerate}

\noindent
\paragraph{Joint distribution.}
The restricted Boltzmann machine is described by a Boltzmann distribution
\begin{align}
	P_{rbm}(\mathbf{x},\mathbf{h}) = \frac{1}{Z} e^{-\frac{1}{T_0}E(\mathbf{x},\mathbf{h})},
\end{align}
where $Z$ is the normalization constant or partition function, defined as 
\begin{align}
	Z = \int \int e^{-\frac{1}{T_0}E(\mathbf{x},\mathbf{h})} d\mathbf{x} d\mathbf{h}.
\end{align}
It is common to ignore $T_0$ by setting it to one. 

\paragraph{Network Elements, the energy function.}
The function $E(\mathbf{x},\mathbf{h})$ gives the \textbf{energy} of a
configuration (pair of vectors) $(\mathbf{x}, \mathbf{h})$. The lower
the energy of a configuration, the higher the probability of it. This
function also depends on the parameters $\mathbf{a}$, $\mathbf{b}$ and
$W$. Thus, when we adjust them during the learning procedure, we are
adjusting the energy function to best fit our problem.

\paragraph{Defining different types of RBMs.}
There are different variants of RBMs, and the differences lie in the types of visible and hidden units we choose as well as in the implementation of the energy function $E(\mathbf{x},\mathbf{h})$. The connection between the nodes in the two layers is given by the weights $w_{ij}$. 


% --- begin paragraph admon ---
\paragraph{Binary-Binary RBM:}

RBMs were first developed using binary units in both the visible and hidden layer. The corresponding energy function is defined as follows:
\begin{align}
	E(\mathbf{x}, \mathbf{h}) = - \sum_i^M x_i a_i- \sum_j^N b_j h_j - \sum_{i,j}^{M,N} x_i w_{ij} h_j,
\end{align}
where the binary values taken on by the nodes are most commonly 0 and 1.
% --- end paragraph admon ---



% --- begin paragraph admon ---
\paragraph{Gaussian-Binary RBM:}

Another varient is the RBM where the visible units are Gaussian while the hidden units remain binary:
\begin{align}
	E(\mathbf{x}, \mathbf{h}) = \sum_i^M \frac{(x_i - a_i)^2}{2\sigma_i^2} - \sum_j^N b_j h_j - \sum_{i,j}^{M,N} \frac{x_i w_{ij} h_j}{\sigma_i^2}. 
\end{align}
% --- end paragraph admon ---



\begin{enumerate}
\item RBMs are Useful when we model continuous data (i.e., we wish $\mathbf{x}$ to be continuous)

\item Requires a smaller learning rate, since there's no upper bound to the value a component might take in the reconstruction
\end{enumerate}

\noindent
Other types of units include:
\begin{enumerate}
\item Softmax and multinomial units

\item Gaussian visible and hidden units

\item Binomial units

\item Rectified linear units
\end{enumerate}

\noindent
\paragraph{Cost function.}
When working with a training dataset, the most common training approach is maximizing the log-likelihood of the training data. The log likelihood characterizes the log-probability of generating the observed data using our generative model. Using this method our cost function is chosen as the negative log-likelihood. The learning then consists of trying to find parameters that maximize the probability of the dataset, and is known as Maximum Likelihood Estimation (MLE).
Denoting the parameters as $\bm{\theta} = a_1,...,a_M,b_1,...,b_N,w_{11},...,w_{MN}$, the log-likelihood is given by
\begin{align}
	\mathcal{L}(\{ \theta_i \}) &= \langle \text{log} P_\theta(\bm{x}) \rangle_{data} \\
	&= - \langle E(\bm{x}; \{ \theta_i\}) \rangle_{data} - \text{log} Z(\{ \theta_i\}),
\end{align}
where we used that the normalization constant does not depend on the data, $\langle \text{log} Z(\{ \theta_i\}) \rangle = \text{log} Z(\{ \theta_i\})$
Our cost function is the negative log-likelihood, $\mathcal{C}(\{ \theta_i \}) = - \mathcal{L}(\{ \theta_i \})$

\paragraph{Optimization / Training.}
The training procedure of choice often is Stochastic Gradient Descent (SGD). It consists of a series of iterations where we update the parameters according to the equation
\begin{align}
	\bm{\theta}_{k+1} = \bm{\theta}_k - \eta \nabla \mathcal{C} (\bm{\theta}_k)
\end{align}
at each $k$-th iteration. There are a range of variants of the algorithm which aim at making the learning rate $\eta$ more adaptive so the method might be more efficient while remaining stable.

We now need the gradient of the cost function in order to minimize it. We find that
\begin{align}
	\frac{\partial \mathcal{C}(\{ \theta_i\})}{\partial \theta_i}
	&= \langle \frac{\partial E(\bm{x}; \theta_i)}{\partial \theta_i} \rangle_{data}
	+ \frac{\partial \text{log} Z(\{ \theta_i\})}{\partial \theta_i} \\
	&= \langle O_i(\bm{x}) \rangle_{data} - \langle O_i(\bm{x}) \rangle_{model},
\end{align}
where in order to simplify notation we defined the "operator"
\begin{align}
	O_i(\bm{x}) = \frac{\partial E(\bm{x}; \theta_i)}{\partial \theta_i}, 
\end{align}
and used the statistical mechanics relationship between expectation values and the log-partition function:
\begin{align}
	\langle O_i(\bm{x}) \rangle_{model} = \text{Tr} P_\theta(\bm{x})O_i(\bm{x}) = - \frac{\partial \text{log} Z(\{ \theta_i\})}{\partial \theta_i}.
\end{align}

The data-dependent term in the gradient is known as the positive phase
of the gradient, while the model-dependent term is known as the
negative phase of the gradient. The aim of the training is to lower
the energy of configurations that are near observed data points
(increasing their probability), and raising the energy of
configurations that are far from observed data points (decreasing
their probability).

The gradient of the negative log-likelihood cost function of a Binary-Binary RBM is then
\begin{align}
	\frac{\partial \mathcal{C} (w_{ij}, a_i, b_j)}{\partial w_{ij}} =& \langle x_i h_j \rangle_{data} - \langle x_i h_j \rangle_{model} \\
	\frac{\partial \mathcal{C} (w_{ij}, a_i, b_j)}{\partial a_{ij}} =& \langle x_i \rangle_{data} - \langle x_i \rangle_{model} \\
	\frac{\partial \mathcal{C} (w_{ij}, a_i, b_j)}{\partial b_{ij}} =& \langle h_i \rangle_{data} - \langle h_i \rangle_{model}. \\
\end{align}
To get the expectation values with respect to the \emph{data}, we set the visible units to each of the observed samples in the training data, then update the hidden units according to the conditional probability found before. We then average over all samples in the training data to calculate expectation values with respect to the data. 

\paragraph{Kullback-Leibler relative entropy.}
When the goal of the training is to approximate a probability
distribution, as it is in generative modeling, another relevant
measure is the \textbf{Kullback-Leibler divergence}, also known as the
relative entropy or Shannon entropy. It is a non-symmetric measure of the
dissimilarity between two probability density functions $p$ and
$q$. If $p$ is the unkown probability which we approximate with $q$,
we can measure the difference by
\begin{align}
	\text{KL}(p||q) = \int_{-\infty}^{\infty} p (\bm{x}) \log \frac{p(\bm{x})}{q(\bm{x})}  d\bm{x}.
\end{align}

Thus, the Kullback-Leibler divergence between the distribution of the
training data $f(\bm{x})$ and the model distribution $p(\bm{x}|
\bm{\theta})$ is

\begin{align}
	\text{KL} (f(\bm{x})|| p(\bm{x}| \bm{\theta})) =& \int_{-\infty}^{\infty}
	f (\bm{x}) \log \frac{f(\bm{x})}{p(\bm{x}| \bm{\theta})} d\bm{x} \\
	=& \int_{-\infty}^{\infty} f(\bm{x}) \log f(\bm{x}) d\bm{x} - \int_{-\infty}^{\infty} f(\bm{x}) \log
	p(\bm{x}| \bm{\theta}) d\bm{x} \\
	%=& \mathbb{E}_{f(\bm{x})} (\log f(\bm{x})) - \mathbb{E}_{f(\bm{x})} (\log p(\bm{x}| \bm{\theta}))
	=& \langle \log f(\bm{x}) \rangle_{f(\bm{x})} - \langle \log p(\bm{x}| \bm{\theta}) \rangle_{f(\bm{x})} \\
	=& \langle \log f(\bm{x}) \rangle_{data} + \langle E(\bm{x}) \rangle_{data} + \log Z \\
	=& \langle \log f(\bm{x}) \rangle_{data} + \mathcal{C}_{LL} .
\end{align}

The first term is constant with respect to $\bm{\theta}$ since $f(\bm{x})$ is independent of $\bm{\theta}$. Thus the Kullback-Leibler Divergence is minimal when the second term is minimal. The second term is the log-likelihood cost function, hence minimizing the Kullback-Leibler divergence is equivalent to maximizing the log-likelihood.

To further understand generative models it is useful to study the
gradient of the cost function which is needed in order to minimize it
using methods like stochastic gradient descent. 

The partition function is the generating function of
expectation values, in particular there are mathematical relationships
between expectation values and the log-partition function. In this
case we have
\begin{align}
	\langle \frac{ \partial E(\bm{x}; \theta_i) } { \partial \theta_i} \rangle_{model}
	= \int p(\bm{x}| \bm{\theta}) \frac{ \partial E(\bm{x}; \theta_i) } { \partial \theta_i} d\bm{x} 
	= -\frac{\partial \log Z(\theta_i)}{ \partial  \theta_i} .
\end{align}

Here $\langle \cdot \rangle_{model}$ is the expectation value over the model probability distribution $p(\bm{x}| \bm{\theta})$.

\subsection*{Setting up for gradient descent calculations}

Using the previous relationship we can express the gradient of the cost function as

\begin{align}
	\frac{\partial \mathcal{C}_{LL}}{\partial \theta_i}
	=& \langle \frac{ \partial E(\bm{x}; \theta_i) } { \partial \theta_i} \rangle_{data} + \frac{\partial \log Z(\theta_i)}{ \partial  \theta_i} \\
	=& \langle \frac{ \partial E(\bm{x}; \theta_i) } { \partial \theta_i} \rangle_{data} - \langle \frac{ \partial E(\bm{x}; \theta_i) } { \partial \theta_i} \rangle_{model} \\
	%=& \langle O_i(\bm{x}) \rangle_{data} - \langle O_i(\bm{x}) \rangle_{model}
\end{align}

This expression shows that the gradient of the log-likelihood cost
function is a \textbf{difference of moments}, with one calculated from
the data and one calculated from the model. The data-dependent term is
called the \textbf{positive phase} and the model-dependent term is
called the \textbf{negative phase} of the gradient. We see now that
minimizing the cost function results in lowering the energy of
configurations $\bm{x}$ near points in the training data and
increasing the energy of configurations not observed in the training
data. That means we increase the model's probability of configurations
similar to those in the training data.

The gradient of the cost function also demonstrates why gradients of
unsupervised, generative models must be computed differently from for
those of for example FNNs. While the data-dependent expectation value
is easily calculated based on the samples $\bm{x}_i$ in the training
data, we must sample from the model in order to generate samples from
which to caclulate the model-dependent term. We sample from the model
by using MCMC-based methods. We can not sample from the model directly
because the partition function $Z$ is generally intractable.

As in supervised machine learning problems, the goal is also here to
perform well on \textbf{unseen} data, that is to have good
generalization from the training data. The distribution $f(x)$ we
approximate is not the \textbf{true} distribution we wish to estimate,
it is limited to the training data. Hence, in unsupervised training as
well it is important to prevent overfitting to the training data. Thus
it is common to add regularizers to the cost function in the same
manner as we discussed for say linear regression.

\paragraph{Mathematical details.}
Because we are restricted to potential functions which are positive it
is convenient to express them as exponentials, so that

\begin{align}
	\phi_C (\bm{x}_C) = e^{-E_C(\bm{x}_C)}
\end{align}

where $E(\bm{x}_C)$ is called an \emph{energy function}, and the
exponential representation is the \emph{Boltzmann distribution}. The
joint distribution is defined as the product of potentials.

The joint distribution of the random variables is then

\begin{align}
	p(\bm{x}) =& \frac{1}{Z} \prod_C \phi_C (\bm{x}_C) \nonumber \\
	=& \frac{1}{Z} \prod_C e^{-E_C(\bm{x}_C)} \nonumber \\
	=& \frac{1}{Z} e^{-\sum_C E_C(\bm{x}_C)} \nonumber \\
	=& \frac{1}{Z} e^{-E(\bm{x})}.
\end{align} 
\begin{align}
	p_{BM}(\bm{x}, \bm{h}) = \frac{1}{Z_{BM}} e^{-\frac{1}{T}E_{BM}(\bm{x}, \bm{h})} ,
\end{align}

with the partition function 
\begin{align}
	Z_{BM} = \int \int e^{-\frac{1}{T} E_{BM}(\tilde{\bm{x}}, \tilde{\bm{h}})} d\tilde{\bm{x}} d\tilde{\bm{h}} .
\end{align}

$T$ is a physics-inspired parameter named temperature and will be assumed to be 1 unless otherwise stated. The energy function of the Boltzmann machine determines the interactions between the nodes and is defined  

\begin{align}
	E_{BM}(\bm{x}, \bm{h}) =& - \sum_{i, k}^{M, K} a_i^k \alpha_i^k (x_i)
	- \sum_{j, l}^{N, L} b_j^l \beta_j^l (h_j) 
	- \sum_{i,j,k,l}^{M,N,K,L} \alpha_i^k (x_i) w_{ij}^{kl} \beta_j^l (h_j) \nonumber \\
	&- \sum_{i, m=i+1, k}^{M, M, K} \alpha_i^k (x_i) v_{im}^k \alpha_m^k (x_m)
	- \sum_{j,n=j+1,l}^{N,N,L} \beta_j^l (h_j) u_{jn}^l \beta_n^l (h_n).
\end{align}

Here $\alpha_i^k (x_i)$ and $\beta_j^l (h_j)$ are one-dimensional
transfer functions or mappings from the given input value to the
desired feature value. They can be arbitrary functions of the input
variables and are independent of the parameterization (parameters
referring to weight and biases), meaning they are not affected by
training of the model. The indices $k$ and $l$ indicate that there can
be multiple transfer functions per variable.  Furthermore, $a_i^k$ and
$b_j^l$ are the visible and hidden bias. $w_{ij}^{kl}$ are weights of
the \textbf{inter-layer} connection terms which connect visible and
hidden units. $ v_{im}^k$ and $u_{jn}^l$ are weights of the
\textbf{intra-layer} connection terms which connect the visible units
to each other and the hidden units to each other, respectively.

We remove the intra-layer connections by setting $v_{im}$ and $u_{jn}$
to zero. The expression for the energy of the RBM is then

\begin{align}
	E_{RBM}(\bm{x}, \bm{h}) = - \sum_{i, k}^{M, K} a_i^k \alpha_i^k (x_i)
	- \sum_{j, l}^{N, L} b_j^l \beta_j^l (h_j) 
	- \sum_{i,j,k,l}^{M,N,K,L} \alpha_i^k (x_i) w_{ij}^{kl} \beta_j^l (h_j). 
\end{align}
resulting in 
\begin{align}
	P_{RBM} (\bm{x}) =& \int P_{RBM} (\bm{x}, \tilde{\bm{h}})  d \tilde{\bm{h}} \nonumber \\
	=& \frac{1}{Z_{RBM}} \int e^{-E_{RBM} (\bm{x}, \tilde{\bm{h}}) } d\tilde{\bm{h}} \nonumber \\
	=& \frac{1}{Z_{RBM}} \int e^{\sum_{i, k} a_i^k \alpha_i^k (x_i)
	+ \sum_{j, l} b_j^l \beta_j^l (\tilde{h}_j) 
	+ \sum_{i,j,k,l} \alpha_i^k (x_i) w_{ij}^{kl} \beta_j^l (\tilde{h}_j)} 
	d\tilde{\bm{h}} \nonumber \\
	=& \frac{1}{Z_{RBM}} e^{\sum_{i, k} a_i^k \alpha_i^k (x_i)}
	\int \prod_j^N e^{\sum_l b_j^l \beta_j^l (\tilde{h}_j) 
	+ \sum_{i,k,l} \alpha_i^k (x_i) w_{ij}^{kl} \beta_j^l (\tilde{h}_j)} d\tilde{\bm{h}} \nonumber \\
	=& \frac{1}{Z_{RBM}} e^{\sum_{i, k} a_i^k \alpha_i^k (x_i)}
	\biggl( \int e^{\sum_l b_1^l \beta_1^l (\tilde{h}_1) + \sum_{i,k,l} \alpha_i^k (x_i) w_{i1}^{kl} \beta_1^l (\tilde{h}_1)} d \tilde{h}_1 \nonumber \\
	& \times \int e^{\sum_l b_2^l \beta_2^l (\tilde{h}_2) + \sum_{i,k,l} \alpha_i^k (x_i) w_{i2}^{kl} \beta_2^l (\tilde{h}_2)} d \tilde{h}_2 \nonumber \\
	& \times ... \nonumber \\
	& \times \int e^{\sum_l b_N^l \beta_N^l (\tilde{h}_N) + \sum_{i,k,l} \alpha_i^k (x_i) w_{iN}^{kl} \beta_N^l (\tilde{h}_N)} d \tilde{h}_N \biggr) \nonumber \\
	=& \frac{1}{Z_{RBM}} e^{\sum_{i, k} a_i^k \alpha_i^k (x_i)}
	\prod_j^N \int e^{\sum_l b_j^l \beta_j^l (\tilde{h}_j) + \sum_{i,k,l} \alpha_i^k (x_i) w_{ij}^{kl} \beta_j^l (\tilde{h}_j)}  d\tilde{h}_j
\end{align}

Similarly

\begin{align}
	P_{RBM} (\bm{h}) =& \frac{1}{Z_{RBM}} \int e^{-E_{RBM} (\tilde{\bm{x}}, \bm{h})} d\tilde{\bm{x}} \nonumber \\
	=& \frac{1}{Z_{RBM}} e^{\sum_{j, l} b_j^l \beta_j^l (h_j)}
	\prod_i^M \int e^{\sum_k a_i^k \alpha_i^k (\tilde{x}_i)
	+ \sum_{j,k,l} \alpha_i^k (\tilde{x}_i) w_{ij}^{kl} \beta_j^l (h_j)} d\tilde{x}_i
\end{align}

Using Bayes theorem

\begin{align}
	P_{RBM} (\bm{h}|\bm{x}) =& \frac{P_{RBM} (\bm{x}, \bm{h})}{P_{RBM} (\bm{x})} \nonumber \\
	=& \frac{\frac{1}{Z_{RBM}} e^{\sum_{i, k} a_i^k \alpha_i^k (x_i)
	+ \sum_{j, l} b_j^l \beta_j^l (h_j) 
	+ \sum_{i,j,k,l} \alpha_i^k (x_i) w_{ij}^{kl} \beta_j^l (h_j)}}
	{\frac{1}{Z_{RBM}} e^{\sum_{i, k} a_i^k \alpha_i^k (x_i)}
	\prod_j^N \int e^{\sum_l b_j^l \beta_j^l (\tilde{h}_j) + \sum_{i,k,l} \alpha_i^k (x_i) w_{ij}^{kl} \beta_j^l (\tilde{h}_j)}  d\tilde{h}_j} \nonumber \\
	=& \prod_j^N \frac{e^{\sum_l b_j^l \beta_j^l (h_j) + \sum_{i,k,l} \alpha_i^k (x_i) w_{ij}^{kl} \beta_j^l (h_j)} }
	{\int e^{\sum_l b_j^l \beta_j^l (\tilde{h}_j) + \sum_{i,k,l} \alpha_i^k (x_i) w_{ij}^{kl} \beta_j^l (\tilde{h}_j)}  d\tilde{h}_j}
\end{align}

Similarly

\begin{align}
	P_{RBM} (\bm{x}|\bm{h}) =&  \frac{P_{RBM} (\bm{x}, \bm{h})}{P_{RBM} (\bm{h})} \nonumber \\
	=& \prod_i^M \frac{e^{\sum_k a_i^k \alpha_i^k (x_i)
	+ \sum_{j,k,l} \alpha_i^k (x_i) w_{ij}^{kl} \beta_j^l (h_j)}}
	{\int e^{\sum_k a_i^k \alpha_i^k (\tilde{x}_i)
	+ \sum_{j,k,l} \alpha_i^k (\tilde{x}_i) w_{ij}^{kl} \beta_j^l (h_j)} d\tilde{x}_i}
\end{align}

The original RBM had binary visible and hidden nodes. They were
showned to be universal approximators of discrete distributions.
It was also shown that adding hidden units yields
strictly improved modelling power. The common choice of binary values
are 0 and 1. However, in some physics applications, -1 and 1 might be
a more natural choice. We will here use 0 and 1.

\begin{align}
	E_{BB}(\bm{x}, \mathbf{h}) = - \sum_i^M x_i a_i- \sum_j^N b_j h_j - \sum_{i,j}^{M,N} x_i w_{ij} h_j.
\end{align}

\begin{align}
	p_{BB}(\bm{x}, \bm{h}) =& \frac{1}{Z_{BB}} e^{\sum_i^M a_i x_i + \sum_j^N b_j h_j + \sum_{ij}^{M,N} x_i w_{ij} h_j} \\
	=& \frac{1}{Z_{BB}} e^{\bm{x}^T \bm{a} + \bm{b}^T \bm{h} + \bm{x}^T \bm{W} \bm{h}}
\end{align}

with the partition function

\begin{align}
	Z_{BB} = \sum_{\bm{x}, \bm{h}} e^{\bm{x}^T \bm{a} + \bm{b}^T \bm{h} + \bm{x}^T \bm{W} \bm{h}} .
\end{align}

\paragraph{Marginal Probability Density Functions.}
In order to find the probability of any configuration of the visible units we derive the marginal probability density function.

\begin{align}
	p_{BB} (\bm{x}) =& \sum_{\bm{h}} p_{BB} (\bm{x}, \bm{h}) \\
	=& \frac{1}{Z_{BB}} \sum_{\bm{h}} e^{\bm{x}^T \bm{a} + \bm{b}^T \bm{h} + \bm{x}^T \bm{W} \bm{h}} \nonumber \\
	=& \frac{1}{Z_{BB}} e^{\bm{x}^T \bm{a}} \sum_{\bm{h}} e^{\sum_j^N (b_j + \bm{x}^T \bm{w}_{\ast j})h_j} \nonumber \\
	=& \frac{1}{Z_{BB}} e^{\bm{x}^T \bm{a}} \sum_{\bm{h}} \prod_j^N e^{ (b_j + \bm{x}^T \bm{w}_{\ast j})h_j} \nonumber \\
	=& \frac{1}{Z_{BB}} e^{\bm{x}^T \bm{a}} \bigg ( \sum_{h_1} e^{(b_1 + \bm{x}^T \bm{w}_{\ast 1})h_1}
	\times \sum_{h_2} e^{(b_2 + \bm{x}^T \bm{w}_{\ast 2})h_2} \times \nonumber \\
	& ... \times \sum_{h_2} e^{(b_N + \bm{x}^T \bm{w}_{\ast N})h_N} \bigg ) \nonumber \\
	=& \frac{1}{Z_{BB}} e^{\bm{x}^T \bm{a}} \prod_j^N \sum_{h_j} e^{(b_j + \bm{x}^T \bm{w}_{\ast j}) h_j} \nonumber \\
	=& \frac{1}{Z_{BB}} e^{\bm{x}^T \bm{a}} \prod_j^N (1 + e^{b_j + \bm{x}^T \bm{w}_{\ast j}}) .
\end{align}

A similar derivation yields the marginal probability of the hidden units

\begin{align}
	p_{BB} (\bm{h}) = \frac{1}{Z_{BB}} e^{\bm{b}^T \bm{h}} \prod_i^M (1 + e^{a_i + \bm{w}_{i\ast}^T \bm{h}}) .
\end{align}

\paragraph{Conditional Probability Density Functions.}
We derive the probability of the hidden units given the visible units using Bayes' rule

\begin{align}
	p_{BB} (\bm{h}|\bm{x}) =& \frac{p_{BB} (\bm{x}, \bm{h})}{p_{BB} (\bm{x})} \nonumber \\
	=& \frac{ \frac{1}{Z_{BB}}  e^{\bm{x}^T \bm{a} + \bm{b}^T \bm{h} + \bm{x}^T \bm{W} \bm{h}} }
	        {\frac{1}{Z_{BB}} e^{\bm{x}^T \bm{a}} \prod_j^N (1 + e^{b_j + \bm{x}^T \bm{w}_{\ast j}})} \nonumber \\
	=& \frac{  e^{\bm{x}^T \bm{a}} e^{ \sum_j^N (b_j + \bm{x}^T \bm{w}_{\ast j} ) h_j} }
	        { e^{\bm{x}^T \bm{a}} \prod_j^N (1 + e^{b_j + \bm{x}^T \bm{w}_{\ast j}})} \nonumber \\
	=& \prod_j^N \frac{ e^{(b_j + \bm{x}^T \bm{w}_{\ast j} ) h_j}  }
	{1 + e^{b_j + \bm{x}^T \bm{w}_{\ast j}}} \nonumber \\
	=& \prod_j^N p_{BB} (h_j| \bm{x}) .
\end{align}

From this we find the probability of a hidden unit being "on" or "off":

\begin{align}
	p_{BB} (h_j=1 | \bm{x}) =&   \frac{ e^{(b_j + \bm{x}^T \bm{w}_{\ast j} ) h_j}  }
	{1 + e^{b_j + \bm{x}^T \bm{w}_{\ast j}}} \\
	=&  \frac{ e^{(b_j + \bm{x}^T \bm{w}_{\ast j} )}  }
	{1 + e^{b_j + \bm{x}^T \bm{w}_{\ast j}}} \\
	=&  \frac{ 1 }{1 + e^{-(b_j + \bm{x}^T \bm{w}_{\ast j})} } ,
\end{align}
and

\begin{align}
	p_{BB} (h_j=0 | \bm{x}) =\frac{ 1 }{1 + e^{b_j + \bm{x}^T \bm{w}_{\ast j}} } .
\end{align}

Similarly we have that the conditional probability of the visible units given the hidden are

\begin{align}
	p_{BB} (\bm{x}|\bm{h}) =& \prod_i^M \frac{ e^{ (a_i + \bm{w}_{i\ast}^T \bm{h}) x_i} }{ 1 + e^{a_i + \bm{w}_{i\ast}^T \bm{h}} } \\
	&= \prod_i^M p_{BB} (x_i | \bm{h}) .
\end{align}

\begin{align}
	p_{BB} (x_i=1 | \bm{h}) =& \frac{1}{1 + e^{-(a_i + \bm{w}_{i\ast}^T \bm{h} )}} \\
	p_{BB} (x_i=0 | \bm{h}) =& \frac{1}{1 + e^{a_i + \bm{w}_{i\ast}^T \bm{h} }} .
\end{align}

\paragraph{Gaussian-Binary Restricted Boltzmann Machines.}
Inserting into the expression for $E_{RBM}(\bm{x},\bm{h})$ in equation  results in the energy

\begin{align}
	E_{GB}(\bm{x}, \bm{h}) =& \sum_i^M \frac{(x_i - a_i)^2}{2\sigma_i^2}
	- \sum_j^N b_j h_j 
	-\sum_{ij}^{M,N} \frac{x_i w_{ij} h_j}{\sigma_i^2} \nonumber \\
	=& \vert\vert\frac{\bm{x} -\bm{a}}{2\bm{\sigma}}\vert\vert^2 - \bm{b}^T \bm{h} 
	- (\frac{\bm{x}}{\bm{\sigma}^2})^T \bm{W}\bm{h} . 
\end{align}

\paragraph{Joint Probability Density Function.}
\begin{align}
	p_{GB} (\bm{x}, \bm{h}) =& \frac{1}{Z_{GB}} e^{-\vert\vert\frac{\bm{x} -\bm{a}}{2\bm{\sigma}}\vert\vert^2 + \bm{b}^T \bm{h} 
	+ (\frac{\bm{x}}{\bm{\sigma}^2})^T \bm{W}\bm{h}} \nonumber \\
	=& \frac{1}{Z_{GB}} e^{- \sum_i^M \frac{(x_i - a_i)^2}{2\sigma_i^2}
	+ \sum_j^N b_j h_j 
	+\sum_{ij}^{M,N} \frac{x_i w_{ij} h_j}{\sigma_i^2}} \nonumber \\
	=& \frac{1}{Z_{GB}} \prod_{ij}^{M,N} e^{-\frac{(x_i - a_i)^2}{2\sigma_i^2}
	+ b_j h_j 
	+\frac{x_i w_{ij} h_j}{\sigma_i^2}} ,
\end{align}

with the partition function given by

\begin{align}
	Z_{GB} =& \int \sum_{\tilde{\bm{h}}}^{\tilde{\bm{H}}} e^{-\vert\vert\frac{\tilde{\bm{x}} -\bm{a}}{2\bm{\sigma}}\vert\vert^2 + \bm{b}^T \tilde{\bm{h}} 
	+ (\frac{\tilde{\bm{x}}}{\bm{\sigma}^2})^T \bm{W}\tilde{\bm{h}}} d\tilde{\bm{x}} .
\end{align}

\paragraph{Marginal Probability Density Functions.}
We proceed to find the marginal probability densitites of the
Gaussian-binary RBM. We first marginalize over the binary hidden units
to find $p_{GB} (\bm{x})$

\begin{align}
	p_{GB} (\bm{x}) =& \sum_{\tilde{\bm{h}}}^{\tilde{\bm{H}}} p_{GB} (\bm{x}, \tilde{\bm{h}}) \nonumber \\
	=& \frac{1}{Z_{GB}} \sum_{\tilde{\bm{h}}}^{\tilde{\bm{H}}} 
	e^{-\vert\vert\frac{\bm{x} -\bm{a}}{2\bm{\sigma}}\vert\vert^2 + \bm{b}^T \tilde{\bm{h}} 
	+ (\frac{\bm{x}}{\bm{\sigma}^2})^T \bm{W}\tilde{\bm{h}}} \nonumber \\
	=& \frac{1}{Z_{GB}} e^{-\vert\vert\frac{\bm{x} -\bm{a}}{2\bm{\sigma}}\vert\vert^2}
	\prod_j^N (1 + e^{b_j + (\frac{\bm{x}}{\bm{\sigma}^2})^T \bm{w}_{\ast j}} ) .
\end{align}

We next marginalize over the visible units. This is the first time we
marginalize over continuous values. We rewrite the exponential factor
dependent on $\bm{x}$ as a Gaussian function before we integrate in
the last step.

\begin{align}
	p_{GB} (\bm{h}) =& \int p_{GB} (\tilde{\bm{x}}, \bm{h}) d\tilde{\bm{x}} \nonumber \\
	=& \frac{1}{Z_{GB}} \int e^{-\vert\vert\frac{\tilde{\bm{x}} -\bm{a}}{2\bm{\sigma}}\vert\vert^2 + \bm{b}^T \bm{h} 
	+ (\frac{\tilde{\bm{x}}}{\bm{\sigma}^2})^T \bm{W}\bm{h}} d\tilde{\bm{x}} \nonumber \\
	=& \frac{1}{Z_{GB}} e^{\bm{b}^T \bm{h} } \int \prod_i^M
	e^{- \frac{(\tilde{x}_i - a_i)^2}{2\sigma_i^2} + \frac{\tilde{x}_i \bm{w}_{i\ast}^T \bm{h}}{\sigma_i^2} } d\tilde{\bm{x}} \nonumber \\
	=& \frac{1}{Z_{GB}} e^{\bm{b}^T \bm{h} }
	\biggl( \int e^{- \frac{(\tilde{x}_1 - a_1)^2}{2\sigma_1^2} + \frac{\tilde{x}_1 \bm{w}_{1\ast}^T \bm{h}}{\sigma_1^2} } d\tilde{x}_1 \nonumber \\
	& \times \int e^{- \frac{(\tilde{x}_2 - a_2)^2}{2\sigma_2^2} + \frac{\tilde{x}_2 \bm{w}_{2\ast}^T \bm{h}}{\sigma_2^2} } d\tilde{x}_2 \nonumber \\
	& \times ... \nonumber \\
	&\times \int e^{- \frac{(\tilde{x}_M - a_M)^2}{2\sigma_M^2} + \frac{\tilde{x}_M \bm{w}_{M\ast}^T \bm{h}}{\sigma_M^2} } d\tilde{x}_M \biggr) \nonumber \\
	=& \frac{1}{Z_{GB}} e^{\bm{b}^T \bm{h}} \prod_i^M
	\int e^{- \frac{(\tilde{x}_i - a_i)^2 - 2\tilde{x}_i \bm{w}_{i\ast}^T \bm{h}}{2\sigma_i^2} } d\tilde{x}_i \nonumber \\
	=& \frac{1}{Z_{GB}} e^{\bm{b}^T \bm{h}} \prod_i^M
	\int e^{- \frac{\tilde{x}_i^2 - 2\tilde{x}_i(a_i + \tilde{x}_i \bm{w}_{i\ast}^T \bm{h}) + a_i^2}{2\sigma_i^2} } d\tilde{x}_i \nonumber \\
	=& \frac{1}{Z_{GB}} e^{\bm{b}^T \bm{h}} \prod_i^M
	\int e^{- \frac{\tilde{x}_i^2 - 2\tilde{x}_i(a_i + \bm{w}_{i\ast}^T \bm{h}) + (a_i + \bm{w}_{i\ast}^T \bm{h})^2 - (a_i + \bm{w}_{i\ast}^T \bm{h})^2 + a_i^2}{2\sigma_i^2} } d\tilde{x}_i \nonumber \\
	=& \frac{1}{Z_{GB}} e^{\bm{b}^T \bm{h}} \prod_i^M
	\int e^{- \frac{(\tilde{x}_i - (a_i + \bm{w}_{i\ast}^T \bm{h}))^2 - a_i^2 -2a_i \bm{w}_{i\ast}^T \bm{h} - (\bm{w}_{i\ast}^T \bm{h})^2 + a_i^2}{2\sigma_i^2} } d\tilde{x}_i \nonumber \\
	=& \frac{1}{Z_{GB}} e^{\bm{b}^T \bm{h}} \prod_i^M
	e^{\frac{2a_i \bm{w}_{i\ast}^T \bm{h} +(\bm{w}_{i\ast}^T \bm{h})^2 }{2\sigma_i^2}}
	\int e^{- \frac{(\tilde{x}_i - a_i - \bm{w}_{i\ast}^T \bm{h})^2}{2\sigma_i^2}}
	d\tilde{x}_i \nonumber \\
	=& \frac{1}{Z_{GB}} e^{\bm{b}^T \bm{h}} \prod_i^M
	\sqrt{2\pi \sigma_i^2}
	e^{\frac{2a_i \bm{w}_{i\ast}^T \bm{h} +(\bm{w}_{i\ast}^T \bm{h})^2 }{2\sigma_i^2}} .
\end{align}

\paragraph{Conditional Probability Density Functions.}
We finish by deriving the conditional probabilities.
\begin{align}
	p_{GB} (\bm{h}| \bm{x}) =& \frac{p_{GB} (\bm{x}, \bm{h})}{p_{GB} (\bm{x})} \nonumber \\
	=& \frac{\frac{1}{Z_{GB}} e^{-\vert\vert\frac{\bm{x} -\bm{a}}{2\bm{\sigma}}\vert\vert^2 + \bm{b}^T \bm{h} 
	+ (\frac{\bm{x}}{\bm{\sigma}^2})^T \bm{W}\bm{h}}}
	{\frac{1}{Z_{GB}} e^{-\vert\vert\frac{\bm{x} -\bm{a}}{2\bm{\sigma}}\vert\vert^2}
	\prod_j^N (1 + e^{b_j + (\frac{\bm{x}}{\bm{\sigma}^2})^T \bm{w}_{\ast j}} ) }
	\nonumber \\
	=& \prod_j^N \frac{e^{(b_j + (\frac{\bm{x}}{\bm{\sigma}^2})^T \bm{w}_{\ast j})h_j } }
	{1 + e^{b_j + (\frac{\bm{x}}{\bm{\sigma}^2})^T \bm{w}_{\ast j}}} \nonumber \\
	=& \prod_j^N p_{GB} (h_j|\bm{x}).
\end{align}
The conditional probability of a binary hidden unit $h_j$ being on or off again takes the form of a sigmoid function

\begin{align}
	p_{GB} (h_j =1 | \bm{x}) =& \frac{e^{b_j + (\frac{\bm{x}}{\bm{\sigma}^2})^T \bm{w}_{\ast j} } }
	{1 + e^{b_j + (\frac{\bm{x}}{\bm{\sigma}^2})^T \bm{w}_{\ast j}}} \nonumber \\
	=& \frac{1}{1 + e^{-b_j - (\frac{\bm{x}}{\bm{\sigma}^2})^T \bm{w}_{\ast j}}} \\
	p_{GB} (h_j =0 | \bm{x}) =&
	\frac{1}{1 + e^{b_j +(\frac{\bm{x}}{\bm{\sigma}^2})^T \bm{w}_{\ast j}}} .
\end{align}

The conditional probability of the continuous $\bm{x}$ now has another form, however.

\begin{align}
	p_{GB} (\bm{x}|\bm{h})
	=& \frac{p_{GB} (\bm{x}, \bm{h})}{p_{GB} (\bm{h})} \nonumber \\
	=& \frac{\frac{1}{Z_{GB}} e^{-\vert\vert\frac{\bm{x} -\bm{a}}{2\bm{\sigma}}\vert\vert^2 + \bm{b}^T \bm{h} 
	+ (\frac{\bm{x}}{\bm{\sigma}^2})^T \bm{W}\bm{h}}}
	{\frac{1}{Z_{GB}} e^{\bm{b}^T \bm{h}} \prod_i^M
	\sqrt{2\pi \sigma_i^2}
	e^{\frac{2a_i \bm{w}_{i\ast}^T \bm{h} +(\bm{w}_{i\ast}^T \bm{h})^2 }{2\sigma_i^2}}}
	\nonumber \\
	=& \prod_i^M \frac{1}{\sqrt{2\pi \sigma_i^2}}
	\frac{e^{- \frac{(x_i - a_i)^2}{2\sigma_i^2} + \frac{x_i \bm{w}_{i\ast}^T \bm{h}}{2\sigma_i^2} }}
	{e^{\frac{2a_i \bm{w}_{i\ast}^T \bm{h} +(\bm{w}_{i\ast}^T \bm{h})^2 }{2\sigma_i^2}}}
	\nonumber \\
	=& \prod_i^M \frac{1}{\sqrt{2\pi \sigma_i^2}}
	\frac{e^{-\frac{x_i^2 - 2a_i x_i + a_i^2 - 2x_i \bm{w}_{i\ast}^T\bm{h} }{2\sigma_i^2} } }
	{e^{\frac{2a_i \bm{w}_{i\ast}^T \bm{h} +(\bm{w}_{i\ast}^T \bm{h})^2 }{2\sigma_i^2}}}
	\nonumber \\
	=& \prod_i^M \frac{1}{\sqrt{2\pi \sigma_i^2}}
	e^{- \frac{x_i^2 - 2a_i x_i + a_i^2 - 2x_i \bm{w}_{i\ast}^T\bm{h}
	+ 2a_i \bm{w}_{i\ast}^T \bm{h} +(\bm{w}_{i\ast}^T \bm{h})^2}
	{2\sigma_i^2} }
	\nonumber \\
	=& \prod_i^M \frac{1}{\sqrt{2\pi \sigma_i^2}}
	e^{ - \frac{(x_i - b_i - \bm{w}_{i\ast}^T \bm{h})^2}{2\sigma_i^2}} \nonumber \\
	=& \prod_i^M \mathcal{N}
	(x_i | b_i + \bm{w}_{i\ast}^T \bm{h}, \sigma_i^2) \\
	\Rightarrow p_{GB} (x_i|\bm{h}) =& \mathcal{N}
	(x_i | b_i + \bm{w}_{i\ast}^T \bm{h}, \sigma_i^2) .
\end{align}

The form of these conditional probabilities explains the name
"Gaussian" and the form of the Gaussian-binary energy function. We see
that the conditional probability of $x_i$ given $\bm{h}$ is a normal
distribution with mean $b_i + \bm{w}_{i\ast}^T \bm{h}$ and variance
$\sigma_i^2$.

\section*{Generative Models}
\textbf{Generative models} describe a class of statistical models that are a contrast
to \textbf{discriminative models}. Informally we say that generative models can
generate new data instances while discriminative models discriminate between
different kinds of data instances. A generative model could generate new photos
of animals that look like 'real' animals while a discriminative model could tell
a dog from a cat. More formally, given a data set $x$ and a set of labels /
targets $y$. Generative models capture the joint probability $p(x, y)$, or
just $p(x)$ if there are no labels, while discriminative models capture the
conditional probability $p(y | x)$. Discriminative models generally try to draw
boundaries in the data space (often high dimensional), while generative models
try to model how data is placed throughout the space.

\subsection*{Generative Adversarial Networks}
\textbf{Generative Adversarial Networks} are a type of unsupervised machine learning
algorithm proposed by \href{{https://arxiv.org/pdf/1406.2661.pdf}}{Goodfellow et. al}
in 2014 (Read the paper first it's only 6 pages). The simplest formulation of
the model is based on a game theoretic approach, \emph{zero sum game}, where we pit
two neural networks against one another. We define two rival networks, one
generator $g$, and one discriminator $d$. The generator directly produces
samples
\begin{equation}
    x = g(z; \theta^{(g)})
\end{equation}
The discriminator attempts to distinguish between samples drawn from the
training data and samples drawn from the generator. In other words, it tries to
tell the difference between the fake data produced by $g$ and the actual data
samples we want to do prediction on. The discriminator outputs a probability
value given by

\begin{equation}
    d(x; \theta^{(d)})
\end{equation}

indicating the probability that $x$ is a real training example rather than a
fake sample the generator has generated. The simplest way to formulate the
learning process in a generative adversarial network is a zero-sum game, in
which a function

\begin{equation}
    v(\theta^{(g)}, \theta^{(d)})
\end{equation}

determines the reward for the discriminator, while the generator gets the
conjugate reward

\begin{equation}
    -v(\theta^{(g)}, \theta^{(d)})
\end{equation}

During learning both of the networks maximize their own reward function, so that
the generator gets better and better at tricking the discriminator, while the
discriminator gets better and better at telling the difference between the fake
and real data. The generator and discriminator alternate on which one trains at
one time (i.e.~for one epoch). In other words, we keep the generator constant
and train the discriminator, then we keep the discriminator constant to train
the generator and repeat. It is this back and forth dynamic which lets GANs
tackle otherwise intractable generative problems. As the generator improves with
 training, the discriminator's performance gets worse because it cannot easily
 tell the difference between real and fake. If the generator ends up succeeding
 perfectly, the the discriminator will do no better than random guessing i.e.
 50\%. This progression in the training poses a problem for the convergence
 criteria for GANs. The discriminator feedback gets less meaningful over time,
 if we continue training after this point then the generator is effectively
 training on junk data which can undo the learning up to that point. Therefore,
 we stop training when the discriminator starts outputting $1/2$ everywhere.
 At convergence we have

\begin{equation}
    g^* = \underset{g}{\mathrm{argmin}}\hspace{2pt}
          \underset{d}{\mathrm{max}}v(\theta^{(g)}, \theta^{(d)})
\end{equation}
The default choice for $v$ is
\begin{equation}
    v(\theta^{(g)}, \theta^{(d)}) = \mathbb{E}_{x\sim p_\mathrm{data}}\log d(x)
                                  + \mathbb{E}_{x\sim p_\mathrm{model}}
                                  \log (1 - d(x))
\end{equation}
The main motivation for the design of GANs is that the learning process requires
neither approximate inference (variational autoencoders for example) nor
approximation of a partition function. In the case where
\begin{equation}
    \underset{d}{\mathrm{max}}v(\theta^{(g)}, \theta^{(d)})
\end{equation}
is convex in $\theta^{(g)} then the procedure is guaranteed to converge and is
asymptotically consistent
( \href{{https://arxiv.org/pdf/1804.09139.pdf}}{Seth Lloyd on QuGANs}  ). This is in
general not the case and it is possible to get situations where the training
process never converges because the generator and discriminator chase one
another around in the parameter space indefinitely. A much deeper discussion on
the currently open research problem of GAN convergence is available
\href{{https://www.deeplearningbook.org/contents/generative_models.html}}{here}. To
anyone interested in learning more about GANs it is a highly recommended read.
Direct quote: "In this best-performing formulation, the generator aims to
increase the log probability that the discriminator makes a mistake, rather than
aiming to decrease the log probability that the discriminator makes the correct
prediction." \href{{https://arxiv.org/abs/1701.00160}}{Another interesting read}

\subsection*{Writing Our First Generative Adversarial Network}
Let us now move on to actually implementing a GAN in tensorflow. We will study
the performance of our GAN on the MNIST dataset. This code is based on and
adapted from the
\href{{https://www.tensorflow.org/tutorials/generative/dcgan}}{google tutorial}

First we import our libraries









\begin{minted}[fontsize=\fontsize{9pt}{9pt},linenos=false,mathescape,baselinestretch=1.0,fontfamily=tt,xleftmargin=7mm]{python}
import os
import time
import numpy as np
import tensorflow as tf
import matplotlib.pyplot as plt
from tensorflow.keras import layers
from tensorflow.keras.utils import plot_model

\end{minted}


Next we define our hyperparameters and import our data the usual way

















\begin{minted}[fontsize=\fontsize{9pt}{9pt},linenos=false,mathescape,baselinestretch=1.0,fontfamily=tt,xleftmargin=7mm]{python}
BUFFER_SIZE = 60000
BATCH_SIZE = 256
EPOCHS = 30

data = tf.keras.datasets.mnist.load_data()
(train_images, train_labels), (test_images, test_labels) = data
train_images = np.reshape(train_images, (train_images.shape[0],
                                         28,
                                         28,
                                         1)).astype('float32')

# we normalize between -1 and 1
train_images = (train_images - 127.5) / 127.5
training_dataset = tf.data.Dataset.from_tensor_slices(
                      train_images).shuffle(BUFFER_SIZE).batch(BATCH_SIZE)

\end{minted}


Let's have a quick look




\begin{minted}[fontsize=\fontsize{9pt}{9pt},linenos=false,mathescape,baselinestretch=1.0,fontfamily=tt,xleftmargin=7mm]{python}
plt.imshow(train_images[0], cmap='Greys')
plt.show()

\end{minted}


Now we define our two models. This is where the 'magic' happens. There are a
huge amount of possible formulations for both models. A lot of engineering and
trial and error can be done here to try to produce better performing models. For
more advanced GANs this is by far the step where you can 'make or break' a
model.

We start with the generator. As stated in the introductory text the generator
$g$ upsamples from a random sample to the shape of what we want to predict. In
our case we are trying to predict MNIST images ($28\times 28$ pixels).




























































\begin{minted}[fontsize=\fontsize{9pt}{9pt},linenos=false,mathescape,baselinestretch=1.0,fontfamily=tt,xleftmargin=7mm]{python}
def generator_model():
    """
    The generator uses upsampling layers tf.keras.layers.Conv2DTranspose() to
    produce an image from a random seed. We start with a Dense layer taking this
    random sample as an input and subsequently upsample through multiple
    convolutional layers.
    """

    # we define our model
    model = tf.keras.Sequential()


    # adding our input layer. Dense means that every neuron is connected and
    # the input shape is the shape of our random noise. The units need to match
    # in some sense the upsampling strides to reach our desired output shape.
    # we are using 100 random numbers as our seed
    model.add(layers.Dense(units=7*7*BATCH_SIZE,
                           use_bias=False,
                           input_shape=(100, )))
    # we normalize the output form the Dense layer
    model.add(layers.BatchNormalization())
    # and add an activation function to our 'layer'. LeakyReLU avoids vanishing
    # gradient problem
    model.add(layers.LeakyReLU())
    model.add(layers.Reshape((7, 7, BATCH_SIZE)))
    assert model.output_shape == (None, 7, 7, BATCH_SIZE)
    # even though we just added four keras layers we think of everything above
    # as 'one' layer

    # next we add our upscaling convolutional layers
    model.add(layers.Conv2DTranspose(filters=128,
                                     kernel_size=(5, 5),
                                     strides=(1, 1),
                                     padding='same',
                                     use_bias=False))
    model.add(layers.BatchNormalization())
    model.add(layers.LeakyReLU())
    assert model.output_shape == (None, 7, 7, 128)

    model.add(layers.Conv2DTranspose(filters=64,
                                     kernel_size=(5, 5),
                                     strides=(2, 2),
                                     padding='same',
                                     use_bias=False))
    model.add(layers.BatchNormalization())
    model.add(layers.LeakyReLU())
    assert model.output_shape == (None, 14, 14, 64)

    model.add(layers.Conv2DTranspose(filters=1,
                                     kernel_size=(5, 5),
                                     strides=(2, 2),
                                     padding='same',
                                     use_bias=False,
                                     activation='tanh'))
    assert model.output_shape == (None, 28, 28, 1)

    return model


\end{minted}


And there we have our 'simple' generator model. Now we move on to defining our
discriminator model $d$, which is a convolutional neural network based image
classifier.































\begin{minted}[fontsize=\fontsize{9pt}{9pt},linenos=false,mathescape,baselinestretch=1.0,fontfamily=tt,xleftmargin=7mm]{python}
def discriminator_model():
    """
    The discriminator is a convolutional neural network based image classifier
    """

    # we define our model
    model = tf.keras.Sequential()
    model.add(layers.Conv2D(filters=64,
                            kernel_size=(5, 5),
                            strides=(2, 2),
                            padding='same',
                            input_shape=[28, 28, 1]))
    model.add(layers.LeakyReLU())
    # adding a dropout layer as you do in conv-nets
    model.add(layers.Dropout(0.3))


    model.add(layers.Conv2D(filters=128,
                            kernel_size=(5, 5),
                            strides=(2, 2),
                            padding='same'))
    model.add(layers.LeakyReLU())
    # adding a dropout layer as you do in conv-nets
    model.add(layers.Dropout(0.3))

    model.add(layers.Flatten())
    model.add(layers.Dense(1))

    return model

\end{minted}


Let us take a look at our models. \textbf{Note}: double click images for bigger view.




\begin{minted}[fontsize=\fontsize{9pt}{9pt},linenos=false,mathescape,baselinestretch=1.0,fontfamily=tt,xleftmargin=7mm]{python}
generator = generator_model()
plot_model(generator, show_shapes=True, rankdir='LR')

\end{minted}





\begin{minted}[fontsize=\fontsize{9pt}{9pt},linenos=false,mathescape,baselinestretch=1.0,fontfamily=tt,xleftmargin=7mm]{python}
discriminator = discriminator_model()
plot_model(discriminator, show_shapes=True, rankdir='LR')

\end{minted}


Next we need a few helper objects we will use in training





\begin{minted}[fontsize=\fontsize{9pt}{9pt},linenos=false,mathescape,baselinestretch=1.0,fontfamily=tt,xleftmargin=7mm]{python}
cross_entropy = tf.keras.losses.BinaryCrossentropy(from_logits=True)
generator_optimizer = tf.keras.optimizers.Adam(1e-4)
discriminator_optimizer = tf.keras.optimizers.Adam(1e-4)

\end{minted}


The first object, \emph{cross_entropy} is our loss function and the two others are
our optimizers. Notice we use the same learning rate for both $g$ and $d$. This
is because they need to improve their accuracy at approximately equal speeds to
get convergence (not necessarily exactly equal). Now we define our loss
functions






\begin{minted}[fontsize=\fontsize{9pt}{9pt},linenos=false,mathescape,baselinestretch=1.0,fontfamily=tt,xleftmargin=7mm]{python}
def generator_loss(fake_output):
    loss = cross_entropy(tf.ones_like(fake_output), fake_output)

    return loss

\end{minted}









\begin{minted}[fontsize=\fontsize{9pt}{9pt},linenos=false,mathescape,baselinestretch=1.0,fontfamily=tt,xleftmargin=7mm]{python}
def discriminator_loss(real_output, fake_output):
    real_loss = cross_entropy(tf.ones_like(real_output), real_output)
    fake_loss = cross_entropy(tf.zeros_liks(fake_output), fake_output)
    total_loss = real_loss + fake_loss

    return total_loss

\end{minted}


Next we define a kind of seed to help us compare the learning process over
multiple training epochs.





\begin{minted}[fontsize=\fontsize{9pt}{9pt},linenos=false,mathescape,baselinestretch=1.0,fontfamily=tt,xleftmargin=7mm]{python}
noise_dimension = 100
n_examples_to_generate = 16
seed_images = tf.random.normal([n_examples_to_generate, noise_dimension])

\end{minted}


Now we have everything we need to define our training step, which we will apply
for every step in our training loop. Notice the @tf.function flag signifying
that the function is tensorflow 'compiled'. Removing this flag doubles the
computation time.

























\begin{minted}[fontsize=\fontsize{9pt}{9pt},linenos=false,mathescape,baselinestretch=1.0,fontfamily=tt,xleftmargin=7mm]{python}
@tf.function
def train_step(images):
    noise = tf.random.normal([BATCH_SIZE, noise_dimension])

    with tf.GradientTape() as gen_tape, tf.GradientTape() as disc_tape:
        generated_images = generator(noise, training=True)

        real_output = discriminator(images, training=True)
        fake_output = discriminator(generated_images, training=True)

        gen_loss = generator_loss(fake_output)
        disc_loss = discriminator_loss(real_output, fake_output)

    gradients_of_generator = gen_tape.gradient(gen_loss,
                                            generator.trainable_variables)
    gradients_of_discriminator = disc_tape.gradient(disc_loss,
                                            discriminator.trainable_variables)
    generator_optimizer.apply_gradients(zip(gradients_of_generator,
                                            generator.trainable_variables))
    discriminator_optimizer.apply_gradients(zip(gradients_of_discriminator,
                                            discriminator.trainable_variables))

    return gen_loss, disc_loss

\end{minted}


Next we define a helper function to produce an output over our training epochs
to see the predictive progression of our generator model. \textbf{Note}: I am including
this code here, but comment it out in the training loop.















\begin{minted}[fontsize=\fontsize{9pt}{9pt},linenos=false,mathescape,baselinestretch=1.0,fontfamily=tt,xleftmargin=7mm]{python}
def generate_and_save_images(model, epoch, test_input):
    # we're making inferences here
    predictions = model(test_input, training=False)

    fig = plt.figure(figsize=(4, 4))

    for i in range(predictions.shape[0]):
        plt.subplot(4, 4, i+1)
        plt.imshow(predictions[i, :, :, 0] * 127.5 + 127.5, cmap='gray')
        plt.axis('off')

    plt.savefig(f'./images_from_seed_images/image_at_epoch_{str(epoch).zfill(3)}.png')
    plt.close()
    #plt.show()

\end{minted}


Setting up checkpoints to periodically save our model during training so that
everything is not lost even if the program were to somehow terminate while
training.









\begin{minted}[fontsize=\fontsize{9pt}{9pt},linenos=false,mathescape,baselinestretch=1.0,fontfamily=tt,xleftmargin=7mm]{python}
# Setting up checkpoints to save model during training
checkpoint_dir = './training_checkpoints'
checkpoint_prefix = os.path.join(checkpoint_dir, 'ckpt')
checkpoint = tf.train.Checkpoint(generator_optimizer=generator_optimizer,
                            discriminator_optimizer=discriminator_optimizer,
                            generator=generator,
                            discriminator=discriminator)

\end{minted}


Now we define our training loop































\begin{minted}[fontsize=\fontsize{9pt}{9pt},linenos=false,mathescape,baselinestretch=1.0,fontfamily=tt,xleftmargin=7mm]{python}
def train(dataset, epochs):
    generator_loss_list = []
    discriminator_loss_list = []

    for epoch in range(epochs):
        start = time.time()

        for image_batch in dataset:
            gen_loss, disc_loss = train_step(image_batch)
            generator_loss_list.append(gen_loss.numpy())
            discriminator_loss_list.append(disc_loss.numpy())

        #generate_and_save_images(generator, epoch + 1, seed_images)

        if (epoch + 1) % 15 == 0:
            checkpoint.save(file_prefix=checkpoint_prefix)

        print(f'Time for epoch {epoch} is {time.time() - start}')

    #generate_and_save_images(generator, epochs, seed_images)

    loss_file = './data/lossfile.txt'
    with open(loss_file, 'w') as outfile:
        outfile.write(str(generator_loss_list))
        outfile.write('\n')
        outfile.write('\n')
        outfile.write(str(discriminator_loss_list))
        outfile.write('\n')
        outfile.write('\n')

\end{minted}


To train simply call this function. \textbf{Warning}: this might take a long time so
there is a folder of a pretrained network already included in the repository.



\begin{minted}[fontsize=\fontsize{9pt}{9pt},linenos=false,mathescape,baselinestretch=1.0,fontfamily=tt,xleftmargin=7mm]{python}
train(train_dataset, EPOCHS)

\end{minted}


And here is the result of training our model for 100 epochs


\begin{doconce:movie}
\refstepcounter{doconce:movie:counter}
\begin{quote}
% link to external viewer
Movie \arabic{doconce:movie:counter}:  \href{run:images_from_seed_images/generation.gif}{\nolinkurl{images_from_seed_images/generation.gif}}
\end{quote}
\end{doconce:movie}


Now to avoid having to train and everything, which will take a while depending
on your computer setup we now load in the model which produced the above gif.








\begin{minted}[fontsize=\fontsize{9pt}{9pt},linenos=false,mathescape,baselinestretch=1.0,fontfamily=tt,xleftmargin=7mm]{python}
checkpoint.restore(tf.train.latest_checkpoint(checkpoint_dir))
restored_generator = checkpoint.generator
restored_discriminator = checkpoint.discriminator

print(restored_generator)
print(restored_discriminator)

\end{minted}


\subsection*{Exploring the Latent Space}

So we have successfully loaded in our latest model. Let us now play around a bit
and see what kind of things we can learn about this model. Our generator takes
an array of 100 numbers. One idea can be to try to systematically change our
input. Let us try and see what we get



















\begin{minted}[fontsize=\fontsize{9pt}{9pt},linenos=false,mathescape,baselinestretch=1.0,fontfamily=tt,xleftmargin=7mm]{python}
def generate_latent_points(number=100, scale_means=1, scale_stds=1):
    latent_dim = 100
    means = scale_means * tf.linspace(-1, 1, num=latent_dim)
    stds = scale_stds * tf.linspace(-1, 1, num=latent_dim)
    latent_space_value_range = tf.random.normal([number, number],
                                                means,
                                                stds,
                                                dtype=tf.float64)

    return latent_space_value_range

def generate_images(latent_points):
    # notice we set training to false because we are making inferences
    generated_images = restored_generator(latent_space_value_range,
                                          training=False)

    return generated_images

\end{minted}














\begin{minted}[fontsize=\fontsize{9pt}{9pt},linenos=false,mathescape,baselinestretch=1.0,fontfamily=tt,xleftmargin=7mm]{python}
def plot_result(generated_images, number):
    # obviously this assumes sqrt number is an int
    fig, axs = plt.subplots(int(np.sqrt(number)), int(np.sqrt(number)),
                            figsize=(10, 10))

    for i in range(int(np.sqrt(number))):
        for j in range(int(np.sqrt(number))):
            axs[i, j].imshow(generated_images[i*j], cmap='Greys')
            axs[i, j].axis('off')

    plt.show()

\end{minted}





\begin{minted}[fontsize=\fontsize{9pt}{9pt},linenos=false,mathescape,baselinestretch=1.0,fontfamily=tt,xleftmargin=7mm]{python}
generated_images = generate_images(generate_latent_points())
plot_result(generated_images, number)

\end{minted}


Interesting! We see that the generator generates images that look like MNIST
numbers: $1, 4, 7, 9$. Let's try to tweak it a bit more to see if we are able
to generate a similar plot where we generate every MNIST number. Let us now try
to 'move' a bit around in the latent space. \textbf{Note}: decrease the plot number if
these following cells take too long to run on your computer.


















\begin{minted}[fontsize=\fontsize{9pt}{9pt},linenos=false,mathescape,baselinestretch=1.0,fontfamily=tt,xleftmargin=7mm]{python}
plot_number = 225

generated_images = generate_images(generate_latent_points(number=plot_number,
                                                          scale_means=5,
                                                          scale_stds=1))
plot_result(generated_images, plot_number)

generated_images = generate_images(generate_latent_points(number=plot_number,
                                                          scale_means=-5,
                                                          scale_stds=1))
plot_result(generated_images, plot_number)

generated_images = generate_images(generate_latent_points(number=plot_number,
                                                          scale_means=1,
                                                          scale_stds=5))
plot_result(generated_images, plot_number)

\end{minted}


Again, we have found something interesting. \emph{Moving} around using our means
takes us from digit to digit, while \emph{moving} around using our standard
deviations seem to increase the number of different digits! In the last image
above, we can barely make out every MNIST digit. Let us make on last plot using
this information by upping the standard deviation of our Gaussian noises.






\begin{minted}[fontsize=\fontsize{9pt}{9pt},linenos=false,mathescape,baselinestretch=1.0,fontfamily=tt,xleftmargin=7mm]{python}
plot_number = 400
generated_images = generate_images(generate_latent_points(number=plot_number,
                                                          scale_means=1,
                                                          scale_stds=10))

\end{minted}

A pretty cool result! We see that our generator indeed has learned a
distribution which qualitatively looks a whole lot like the MNIST dataset.

\subsection*{Interpolating Between MNIST Digits}
Another interesting way to explore the latent space of our generator model is by
interpolating between the MNIST digits. This section is largely based on
"this excellent blogpost": \href{{https://machinelearningmastery.com/how-to-interpolate-and-perform-vector-arithmetic-with-faces-using-a-generative-adversarial-network/}}{\nolinkurl{https://machinelearningmastery.com/how-to-interpolate-and-perform-vector-arithmetic-with-faces-using-a-generative-adversarial-network/}}
by Jason Brownlee.

So let us start

% ------------------- end of main content ---------------

\end{document}

