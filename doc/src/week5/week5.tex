%%
%% Automatically generated file from DocOnce source
%% (https://github.com/doconce/doconce/)
%% doconce format latex week5.do.txt --minted_latex_style=trac --latex_admon=paragraph --no_mako
%%


%-------------------- begin preamble ----------------------

\documentclass[%
oneside,                 % oneside: electronic viewing, twoside: printing
final,                   % draft: marks overfull hboxes, figures with paths
10pt]{article}

\listfiles               %  print all files needed to compile this document

\usepackage{relsize,makeidx,color,setspace,amsmath,amsfonts,amssymb}
\usepackage[table]{xcolor}
\usepackage{bm,ltablex,microtype}

\usepackage[pdftex]{graphicx}

\usepackage{fancyvrb} % packages needed for verbatim environments
\usepackage{minted}
\usemintedstyle{default}

\usepackage[T1]{fontenc}
%\usepackage[latin1]{inputenc}
\usepackage{ucs}
\usepackage[utf8x]{inputenc}

\usepackage{lmodern}         % Latin Modern fonts derived from Computer Modern

% Hyperlinks in PDF:
\definecolor{linkcolor}{rgb}{0,0,0.4}
\usepackage{hyperref}
\hypersetup{
    breaklinks=true,
    colorlinks=true,
    linkcolor=linkcolor,
    urlcolor=linkcolor,
    citecolor=black,
    filecolor=black,
    %filecolor=blue,
    pdfmenubar=true,
    pdftoolbar=true,
    bookmarksdepth=3   % Uncomment (and tweak) for PDF bookmarks with more levels than the TOC
    }
%\hyperbaseurl{}   % hyperlinks are relative to this root

\setcounter{tocdepth}{2}  % levels in table of contents

% --- fancyhdr package for fancy headers ---
\usepackage{fancyhdr}
\fancyhf{} % sets both header and footer to nothing
\renewcommand{\headrulewidth}{0pt}
\fancyfoot[LE,RO]{\thepage}
% Ensure copyright on titlepage (article style) and chapter pages (book style)
\fancypagestyle{plain}{
  \fancyhf{}
  \fancyfoot[C]{{\footnotesize \copyright\ 1999-2023, Morten Hjorth-Jensen. Released under CC Attribution-NonCommercial 4.0 license}}
%  \renewcommand{\footrulewidth}{0mm}
  \renewcommand{\headrulewidth}{0mm}
}
% Ensure copyright on titlepages with \thispagestyle{empty}
\fancypagestyle{empty}{
  \fancyhf{}
  \fancyfoot[C]{{\footnotesize \copyright\ 1999-2023, Morten Hjorth-Jensen. Released under CC Attribution-NonCommercial 4.0 license}}
  \renewcommand{\footrulewidth}{0mm}
  \renewcommand{\headrulewidth}{0mm}
}

\pagestyle{fancy}


\usepackage[framemethod=TikZ]{mdframed}

% --- begin definitions of admonition environments ---

% --- end of definitions of admonition environments ---

% prevent orhpans and widows
\clubpenalty = 10000
\widowpenalty = 10000

% --- end of standard preamble for documents ---


% insert custom LaTeX commands...

\raggedbottom
\makeindex
\usepackage[totoc]{idxlayout}   % for index in the toc
\usepackage[nottoc]{tocbibind}  % for references/bibliography in the toc

%-------------------- end preamble ----------------------

\begin{document}

% matching end for #ifdef PREAMBLE

\newcommand{\exercisesection}[1]{\subsection*{#1}}


% ------------------- main content ----------------------



% ----------------- title -------------------------

\thispagestyle{empty}

\begin{center}
{\LARGE\bf
\begin{spacing}{1.25}
February 20-24: Advanced machine learning and data analysis for the physical sciences
\end{spacing}
}
\end{center}

% ----------------- author(s) -------------------------

\begin{center}
{\bf Morten Hjorth-Jensen${}^{1, 2}$} \\ [0mm]
\end{center}

\begin{center}
% List of all institutions:
\centerline{{\small ${}^1$Department of Physics and Center for Computing in Science Education, University of Oslo, Norway}}
\centerline{{\small ${}^2$Department of Physics and Astronomy and Facility for Rare Isotope Beams, Michigan State University, East Lansing, Michigan, USA}}
\end{center}
    
% ----------------- end author(s) -------------------------

% --- begin date ---
\begin{center}
February 20-24, 2023
\end{center}
% --- end date ---

\vspace{1cm}


\subsection*{Plans for February 20-24}


% --- begin paragraph admon ---
\paragraph{}

\begin{itemize}
\item Basics and mathematics of Convolutional  Neural Networks.

\item Discussion of first project with examples

\item \href{{https://youtu.be/}}{Video of lecture}

\item Reading recommendations:
\begin{enumerate}

 \item For neural networks we recommend Goodfellow et al chapters 6 and 7. For CNNs, see Goodfellow et al chapter 9. See also chapter 11 and 12 on practicalities and applications

 \item Reading suggestions for implementation of CNNs: \href{{https://github.com/CompPhysics/MachineLearning/blob/master/doc/Textbooks/TensorflowML.pdf}}{Aurelien Geron's chapter 13}. 
\end{enumerate}

\noindent
\end{itemize}

\noindent
% --- end paragraph admon ---




% --- begin paragraph admon ---
\paragraph{Excellent lectures on CNNs and Neural Networks.}
\begin{itemize}
\item \href{{https://www.youtube.com/playlist?list=PLZHQObOWTQDNU6R1_67000Dx_ZCJB-3pi}}{Video on Deep Learning}

\item \href{{https://www.youtube.com/watch?v=iaSUYvmCekI&ab_channel=AlexanderAmini}}{Video  on Convolutional Neural Networks from MIT}

\item \href{{https://www.youtube.com/watch?v=bNb2fEVKeEo&list=PLC1qU-LWwrF64f4QKQT-Vg5Wr4qEE1Zxk&index=6&ab_channel=StanfordUniversitySchoolofEngineering}}{Video on CNNs from Stanford}
\end{itemize}

\noindent
% --- end paragraph admon ---


m


% --- begin paragraph admon ---
\paragraph{And Lecture material on CNNs.}
\begin{itemize}
\item \href{{https://www.uio.no/studier/emner/matnat/ifi/IN5400/v19/material/week5/in5400_2019_week5_convolutional_nerual_networks.pdf}}{Lectures from IN5400 spring 2019}

\item \href{{https://www.uio.no/studier/emner/matnat/ifi/IN5400/v21/lecture-slides/in5400_2021_w5_lecture_convolutions.pdf}}{Lectures from IN5400 spring 2021}

\item \href{{http://neuralnetworksanddeeplearning.com/chap6.html}}{See also Michael Nielsen's Lectures}
\end{itemize}

\noindent
% --- end paragraph admon ---



% !split
\subsection*{Mathematics of CNNs}

The mathematics of CNNs is based on the mathematical operation of
\textbf{convolution}.  In mathematics (in particular in functional analysis),
convolution is represented by mathematical operations (integration,
summation etc) on two functions in order to produce a third function
that expresses how the shape of one gets modified by the other.
Convolution has a plethora of applications in a variety of
disciplines, spanning from statistics to signal processing, computer
vision, solutions of differential equations,linear algebra,
engineering, and yes, machine learning.

Mathematically, convolution is defined as follows (one-dimensional example):
Let us define a continuous function $y(t)$ given by
\[
y(t) = \int x(a) w(t-a) da,
\]
where $x(a)$ represents a so-called input and $w(t-a)$ is normally called the weight function or kernel.

The above integral is written in  a more compact form as
\[
y(t) = \left(x * w\right)(t).
\]

The discretized version reads
\[
y(t) = \sum_{a=-\infty}^{a=\infty}x(a)w(t-a).
\]
Computing the inverse of the above convolution operations is known as deconvolution and the process is commutative.

How can we use this? And what does it mean? Let us study some familiar examples first.

% !split
\subsection*{Convolution Examples: Polynomial multiplication}

Our first example is that of a multiplication between two polynomials,
which we will rewrite in terms of the mathematics of convolution. In
the final stage, since the problem here is a discrete one, we will
recats the final expression in terms of a matrix-vector
multiplication, where the matrix is a so-called \href{{https://link.springer.com/book/10.1007/978-93-86279-04-0}}{Toeplitz matrix}.

Let us look a the following polynomials to second and third order, respectively:
\[
p(t) = \alpha_0+\alpha_1 t+\alpha_2 t^2,
\]
and
\[
s(t) = \beta_0+\beta_1 t+\beta_2 t^2+\beta_3 t^3.
\]

The polynomial multiplication gives us a new polynomial of degree $5$
\[
z(t) = \delta_0+\delta_1 t+\delta_2 t^2+\delta_3 t^3+\delta_4 t^4+\delta_5 t^5.
\]

% !split
\subsection*{Efficient Polynomial Multiplication}

Computing polynomial products can be implemented efficiently if we rewrite the more brute force multiplications using convolution.
We note first that the new coefficients are given as

\begin{split}
\delta_0=&\alpha_0\beta_0\\
\delta_1=&\alpha_1\beta_0+\alpha_0\beta_1\\
\delta_2=&\alpha_0\beta_2+\alpha_1\beta_1+\alpha_2\beta_0\\
\delta_3=&\alpha_1\beta_2+\alpha_2\beta_1+\alpha_0\beta_3\\
\delta_4=&\alpha_2\beta_2+\alpha_1\beta_3\\
\delta_5=&\alpha_2\beta_3.\\
\end{split}

We note that $\alpha_i=0$ except for $i\in \left\{0,1,2\right\}$ and $\beta_i=0$ except for $i\in\left\{0,1,2,3\right\}$.

We can then rewrite the coefficients $\delta_j$ using a discrete convolution as
\[
\delta_j = \sum_{i=-\infty}^{i=\infty}\alpha_i\beta_{j-i}=(\alpha * \beta)_j,
\]
or as a double sum with restriction $l=i+j$
\[
\delta_l = \sum_{ij}\alpha_i\beta_{j}.
\]

% !split
\subsection*{A more efficient way of coding the above Convolution}

Since we only have a finite number of $\alpha$ and $\beta$ values
which are non-zero, we can rewrite the above convolution expressions
as a matrix-vector multiplication

\[
\bm{\delta}=\begin{bmatrix}\alpha_0 & 0 & 0 & 0 \\
                            \alpha_1 & \alpha_0 & 0 & 0 \\
			    \alpha_2 & \alpha_1 & \alpha_0 & 0 \\
			    0 & \alpha_2 & \alpha_1 & \alpha_0 \\
			    0 & 0 & \alpha_2 & \alpha_1 \\
			    0 & 0 & 0 & \alpha_2
			    \end{bmatrix}\begin{bmatrix} \beta_0 \\ \beta_1 \\ \beta_2 \\ \beta_3\end{bmatrix}.
\]

The process is commutative and we can easily see that we can rewrite the multiplication in terms of  a matrix holding $\beta$ and a vector holding $\alpha$.
In this case we have
\[
\bm{\delta}=\begin{bmatrix}\beta_0 & 0 & 0  \\
                            \beta_1 & \beta_0 & 0  \\
			    \beta_2 & \beta_1 & \beta_0  \\
			    \beta_3 & \beta_2 & \beta_1 \\
			    0 & \beta_3 & \beta_2 \\
			    0 & 0 & \beta_3
			    \end{bmatrix}\begin{bmatrix} \alpha_0 \\ \alpha_1 \\ \alpha_2\end{bmatrix}.
\]

Note that the use of these matrices is for mathematical purposes only
and not implementation purposes.  When implementing the above equation
we do not encode (and allocate memory) the matrices explicitely.  We
rather code the convolutions in the minimal memory footprint that they
require.

The above matrices are examples of so-called \href{{https://link.springer.com/book/10.1007/978-93-86279-04-0}}{Toeplitz
matrices}. A
Toeplitz matrix is a matrix in which each descending diagonal from
left to right is constant. For instance the last matrix, which we
rewrite as
\[
\bm{A}=\begin{bmatrix}a_0 & 0 & 0  \\
                            a_1 & a_0 & 0  \\
			    a_2 & a_1 & a_0  \\
			    a_3 & a_2 & a_1 \\
			    0 & a_3 & a_2 \\
			    0 & 0 & a_3
			    \end{bmatrix},
\]
with elements $a_{ii}=a_{i+1,j+1}=a_{i-j}$ is an example of a Toeplitz
matrix. Such a matrix does not need to be a square matrix.  Toeplitz
matrices are also closely connected with Fourier series discussed
below, because the multiplication operator by a trigonometric
polynomial, compressed to a finite-dimensional space, can be
represented by such a matrix. The example above shows that we can
represent linear convolution as multiplication of a Toeplitz matrix by
a vector.

% !split
\subsection*{Convolution Examples: Principle of Superposition and Periodic Forces (Fourier Transforms)}

For problems with so-called harmonic oscillations, given by for example the following differential equation
\[
m\frac{d^2x}{dt^2}+\eta\frac{dx}{dt}+x(t)=F(t),
\]
where $F(t)$ is an applied external force acting on the system (often
called a driving force), one can use the theory of Fourier
transformations to find the solutions of this type of equations.

If one has several driving forces, $F(t)=\sum_n F_n(t)$, one can find
the particular solution $x_{pn}(t)$ to the above differential equation for each $F_n$. The particular
solution for the entire driving force is then given by a series like

\begin{equation}
x_p(t)=\sum_nx_{pn}(t).
\end{equation}

This is known as the principle of superposition. It only applies when
the homogenous equation is linear. 
Superposition is especially useful when $F(t)$ can be written
as a sum of sinusoidal terms, because the solutions for each
sinusoidal (sine or cosine)  term is analytic. 

Driving forces are often periodic, even when they are not
sinusoidal. Periodicity implies that for some time $t$ our function repeats itself periodically after a period $\tau$, that is

\begin{eqnarray}
F(t+\tau)=F(t). 
\end{eqnarray}

One example of a non-sinusoidal periodic force is a square wave. Many
components in electric circuits are non-linear, for example diodes. This 
makes many wave forms non-sinusoidal even when the circuits are being
driven by purely sinusoidal sources.

% !split
\subsection*{Simple Code Example}

The code here shows a typical example of such a square wave generated
using the functionality included in the \textbf{scipy} Python package. We
have used a period of $\tau=0.2$.



















\begin{minted}[fontsize=\fontsize{9pt}{9pt},linenos=false,mathescape,baselinestretch=1.0,fontfamily=tt,xleftmargin=7mm]{python}
import numpy as np
import math
from scipy import signal
import matplotlib.pyplot as plt

# number of points                                                                                       
n = 500
# start and final times                                                                                  
t0 = 0.0
tn = 1.0
# Period                                                                                                 
t = np.linspace(t0, tn, n, endpoint=False)
SqrSignal = np.zeros(n)
SqrSignal = 1.0+signal.square(2*np.pi*5*t)
plt.plot(t, SqrSignal)
plt.ylim(-0.5, 2.5)
plt.show()

\end{minted}


For the sinusoidal example the
period is $\tau=2\pi/\omega$. However, higher harmonics can also
satisfy the periodicity requirement. In general, any force that
satisfies the periodicity requirement can be expressed as a sum over
harmonics,

\begin{equation}
F(t)=\frac{f_0}{2}+\sum_{n>0} f_n\cos(2n\pi t/\tau)+g_n\sin(2n\pi t/\tau).
\end{equation}

% !split
\subsection*{Wrapping up Fourier transforms}

We can write down the answer for
$x_{pn}(t)$, by substituting $f_n/m$ or $g_n/m$ for $F_0/m$. By
writing each factor $2n\pi t/\tau$ as $n\omega t$, with $\omega\equiv
2\pi/\tau$,

\begin{equation}
\label{eq:fourierdef1}
F(t)=\frac{f_0}{2}+\sum_{n>0}f_n\cos(n\omega t)+g_n\sin(n\omega t).
\end{equation}

The solutions for $x(t)$ then come from replacing $\omega$ with
$n\omega$ for each term in the particular solution,

\begin{eqnarray}
x_p(t)&=&\frac{f_0}{2k}+\sum_{n>0} \alpha_n\cos(n\omega t-\delta_n)+\beta_n\sin(n\omega t-\delta_n),\\
\nonumber
\alpha_n&=&\frac{f_n/m}{\sqrt{((n\omega)^2-\omega_0^2)+4\beta^2n^2\omega^2}},\\
\nonumber
\beta_n&=&\frac{g_n/m}{\sqrt{((n\omega)^2-\omega_0^2)+4\beta^2n^2\omega^2}},\\
\nonumber
\delta_n&=&\tan^{-1}\left(\frac{2\beta n\omega}{\omega_0^2-n^2\omega^2}\right).
\end{eqnarray}

% !split
\subsection*{Finding the Coefficients}

Because the forces have been applied for a long time, any non-zero
damping eliminates the homogenous parts of the solution. We need then 
only consider the particular solution for each $n$.

The problem is considered solved if one can find expressions for the
coefficients $f_n$ and $g_n$, even though the solutions are expressed
as an infinite sum. The coefficients can be extracted from the
function $F(t)$ by

\begin{eqnarray}
\label{eq:fourierdef2}
f_n&=&\frac{2}{\tau}\int_{-\tau/2}^{\tau/2} dt~F(t)\cos(2n\pi t/\tau),\\
\nonumber
g_n&=&\frac{2}{\tau}\int_{-\tau/2}^{\tau/2} dt~F(t)\sin(2n\pi t/\tau).
\end{eqnarray}

To check the consistency of these expressions and to verify
Eq. (\ref{eq:fourierdef2}), one can insert the expansion of $F(t)$ in
Eq. (\ref{eq:fourierdef1}) into the expression for the coefficients in
Eq. (\ref{eq:fourierdef2}) and see whether

\[
f_n=\frac{2}{\tau}\int_{-\tau/2}^{\tau/2} dt~\left\{\frac{f_0}{2}+\sum_{m>0}f_m\cos(m\omega t)+g_m\sin(m\omega t)\right\}\cos(n\omega t).
\]

Immediately, one can throw away all the terms with $g_m$ because they
convolute an even and an odd function. The term with $f_0/2$
disappears because $\cos(n\omega t)$ is equally positive and negative
over the interval and will integrate to zero. For all the terms
$f_m\cos(m\omega t)$ appearing in the sum, one can use angle addition
formulas to see that $\cos(m\omega t)\cos(n\omega
t)=(1/2)(\cos[(m+n)\omega t]+\cos[(m-n)\omega t]$. This will integrate
to zero unless $m=n$. In that case the $m=n$ term gives

\begin{equation}
\int_{-\tau/2}^{\tau/2}dt~\cos^2(m\omega t)=\frac{\tau}{2},
\end{equation}

and

\[
f_n=\frac{2}{\tau}\int_{-\tau/2}^{\tau/2} dt~f_n/2=f_n.
\]

The same method can be used to check for the consistency of $g_n$.

% !split
\subsection*{Final words on Fourier Transforms}

The code here uses the Fourier series applied to a 
square wave signal. The code here
visualizes the various approximations given by Fourier series compared
with a square wave with period $T=0.2$ (dimensionless time), width $0.1$ and max value of the force $F=2$. We
see that when we increase the number of components in the Fourier
series, the Fourier series approximation gets closer and closer to the
square wave signal.































\begin{minted}[fontsize=\fontsize{9pt}{9pt},linenos=false,mathescape,baselinestretch=1.0,fontfamily=tt,xleftmargin=7mm]{python}
import numpy as np
import math
from scipy import signal
import matplotlib.pyplot as plt

# number of points                                                                                       
n = 500
# start and final times                                                                                  
t0 = 0.0
tn = 1.0
# Period                                                                                                 
T =0.2
# Max value of square signal                                                                             
Fmax= 2.0
# Width of signal   
Width = 0.1
t = np.linspace(t0, tn, n, endpoint=False)
SqrSignal = np.zeros(n)
FourierSeriesSignal = np.zeros(n)
SqrSignal = 1.0+signal.square(2*np.pi*5*t+np.pi*Width/T)
a0 = Fmax*Width/T
FourierSeriesSignal = a0
Factor = 2.0*Fmax/np.pi
for i in range(1,500):
    FourierSeriesSignal += Factor/(i)*np.sin(np.pi*i*Width/T)*np.cos(i*t*2*np.pi/T)
plt.plot(t, SqrSignal)
plt.plot(t, FourierSeriesSignal)
plt.ylim(-0.5, 2.5)
plt.show()

\end{minted}


\paragraph{Fourier transforms and convolution.}
We can use Fourier transforms in our studies of convolution as well. To see this, assume we have two functions $f$ and $g$ and their corresponding Fourier transforms $\hat{f}$ and $\hat{g}$. We remind the reader that the Fourier transform reads (say for the function $f$)

\[
\hat{f}(y)=\bm{F}[f(y)]=\frac{1}{2\pi}\int_{-\infty}^{\infty} d\omega \exp{-i\omega y} f(\omega),
\]
and similarly we have
\[
\hat{g}(y)=\bm{F}[g(y)]=\frac{1}{2\pi}\int_{-\infty}^{\infty} d\omega \exp{-i\omega y} g(\omega).
\]
The inverse Fourier transform is given by
\[
\bm{F}^{-1}[g(y)]=\frac{1}{2\pi}\int_{-\infty}^{\infty} d\omega \exp{i\omega y} g(\omega).
\]

The inverse Fourier transform of the product of the two functions $\hat{f}\hat{g}$ can be written as
\[
\bm{F}^{-1}[(\hat{f}\hat{g})(x)]=\frac{1}{2\pi}\int_{-\infty}^{\infty} d\omega \exp{i\omega x} \hat{f}(\omega)\hat{g}(\omega).
\]
We can rewrite the latter as
\[
\bm{F}^{-1}[(\hat{f}\hat{g})(x)]=\int_{-\infty}^{\infty} d\omega \exp{i\omega x} \hat{f}(\omega)\left[\frac{1}{2\pi}\int_{-\infty}^{\infty}g(y)dy \exp{-i\omega y}\right]=\frac{1}{2\pi}\int_{-\infty}^{\infty}dy g(y)\int_{-\infty}^{\infty} d\omega \hat{f}(\omega) \exp{i\omega(x- y)},
\]
which is simply 
\[
\bm{F}^{-1}[(\hat{f}\hat{g})(x)]=\int_{-\infty}^{\infty}dy g(y)f(x-y)=(f*g)(x),
\]
the convolution of the functions $f$ and $g$.

\subsection*{Two-dimensional Objects}

We are now ready to start studying the discrete convolutions relevant for convolutional neural networks.
We often use convolutions over more than one dimension at a time. If
we have a two-dimensional image $I$ as input, we can have a \textbf{filter}
defined by a two-dimensional \textbf{kernel} $K$. This leads to an output $S$

\[
S_(i,j)=(I * K)(i,j) = \sum_m\sum_n I(m,n)K(i-m,j-n).
\]

Convolution is a commutatitave process, which means we can rewrite this equation as
\[
S_(i,j)=(I * K)(i,j) = \sum_m\sum_n I(i-m,j-n)K(m,n).
\]

Normally the latter is more straightforward to implement in  a machine larning library since there is less variation in the range of values of $m$ and $n$.

Many deep learning libraries implement cross-correlation instead of convolution (although it is referred to s convolution)
\[
S_(i,j)=(I * K)(i,j) = \sum_m\sum_n I(i+m,j+n)K(m,n).
\]

% !split
\subsection*{More on Dimensionalities}

In fields like signal processing (and imaging as well), one designs
so-called filters. These filters are defined by the convolutions and
are often hand-crafted. One may specify filters for smoothing, edge
detection, frequency reshaping, and similar operations. However with
neural networks the idea is to automatically learn the filters and use
many of them in conjunction with non-linear operations (activation
functions).

As an example consider a neural network operating on sound sequence
data.  Assume that we an input vector $\bm{x}$ of length $d=10^6$.  We
construct then a neural network with onle hidden layer only with
$10^4$ nodes. This means that we will have a weight matrix with
$10^4\times 10^6=10^{10}$ weights to be determined, together with $10^4$ biases.

Assume furthermore that we have an output layer which is meant to train whether the sound sequence represents a human voice (true) or something else (false).
It means that we have only one output node. But since this output node connects to $10^4$ nodes in the hidden layer, there are in total $10^4$ weights to be determined for the output layer, plus one bias. In total we have

\[
\mathrm{NumberParameters}=10^{10}+10^4+10^4+1 \approx 10^{10},
\]
that is ten billion parameters to determine. 

% !split
\subsection*{Further Dimensionality Remarks}

In today’s architecture one can train such neural networks, however
this is a huge number of parameters for the task at hand. In general,
it is a very wasteful and inefficient use of dense matrices as
parameters. Just as importantly, such trained network parameters are
very specific for the type of input data on which they were trained
and the network is not likely to generalize easily to variations in
the input.

The main principles that justify convolutions is locality of
information and repetion of patterns within the signal. Sound samples
of the input in adjacent spots are much more likely to affect each
other than those that are very far away. Similarly, sounds are
repeated in multiple times in the signal. While slightly simplistic,
reasoning about such a sound example demonstrates this. The same
principles then apply to images and other similar data.

% !split
\subsection*{CNNs in more detail}

Let assume we have an input matrix $I$ of dimensionality $3\times 3$
and a $2\times 2$ filter $W$ given by the following matrices

\[
\bm{I}=\begin{bmatrix}i_{00} & i_{01} & i_{02}  \\
                      i_{10} & i_{11} & i_{12}  \\
	              i_{20} & i_{21} & i_{22} \end{bmatrix},
\]
and 
\[
\bm{W}=\begin{bmatrix}w_{00} & w_{01} \\
	              w_{10} & w_{11}\end{bmatrix}.
\]
We introduce now the hyperparameter $S$ \textbf{stride}. Stride represents how the filter $W$ moves the convolution process on the matrix $I$.
We strongly recommend the repository on \href{{https://github.com/vdumoulin/conv_arithmetic}}{Arithmetic of deep learning by Dumoulin and Visin} 

Here we set the stride equal to $S=1$, which means that, starting with the element $i_{00}$, the filter will act on $2\times 2$ submatrices each time, starting with the upper corner and moving according to the stride value column by column. 

Here we perform the operation
\[
S_(i,j)=(I * W)(i,j) = \sum_m\sum_n I(i-m,j-n)W(m,n),
\]
and obtain
\[
\bm{S}=\begin{bmatrix}i_{00}w_{00}+i_{01}w_{01}+i_{10}w_{10}+i_{11}w_{11} & i_{01}w_{00}+i_{02}w_{01}+i_{11}w_{10}+i_{12}w_{11}  \\
	              i_{10}w_{00}+i_{11}w_{01}+i_{20}w_{10}+i_{21}w_{11} & i_{11}w_{00}+i_{12}w_{01}+i_{21}w_{10}+i_{22}w_{11}\end{bmatrix}.
\]

We can rewrite this operation in terms of a matrix-vector multiplication by defining a new vector where we flatten out the inputs as a vector $\bm{I}'$ of length $9$ and
a matrix $\bm{W}'$ with dimension $4\times 9$ as

\[
\bm{I}'=\begin{bmatrix}i_{00} \\ i_{01} \\ i_{02} \\ i_{10} \\ i_{11} \\ i_{12} \\ i_{20} \\ i_{21} \\ i_{22} \end{bmatrix},
\]

and the new matrix
\[
\bm{W}'=\begin{bmatrix} w_{00} & w_{01} & 0 & w_{10} & w_{11} & 0 & 0 & 0 & 0 \\
                        0  & w_{00} & w_{01} & 0 & w_{10} & w_{11} & 0 & 0 & 0 \\
			0 & 0 & 0 & w_{00} & w_{01} & 0 & w_{10} & w_{11} & 0  \\
                        0 & 0 & 0 & 0 & w_{00} & w_{01} & 0 & w_{10} & w_{11}\end{bmatrix}.
\]

We see easily that performing the matrix-vector multiplication $\bm{W}'\bm{I}'$ is the same as the above convolution with stride $S=1$, that is

\[
S=(\bm{W}*\bm{I}),
\]
is now given by $\bm{W}'\bm{I}'$ which is a vector of length $4$ instead of the originally resulting  $2\times 2$ output matrix.

The collection of kernels/filters $W$ defining a discrete convolution has a shape
corresponding to some permutation of $(n, m, k_1, \ldots, k_N)$, where

\begin{equation*}
\begin{split}
    n &\equiv \text{number of output feature maps},\\
    m &\equiv \text{number of input feature maps},\\
    k_j &\equiv \text{kernel size along axis $j$}.
\end{split}
\end{equation*}

The following properties affect the output size $o_j$ of a convolutional layer
along axis $j$:

\begin{enumerate}
\item $i_j$: input size along axis $j$,

\item $k_j$: kernel/filter size along axis $j$,

\item stride (distance between two consecutive positions of the kernel/filter) along axis $j$,

\item zero padding (number of zeros concatenated at the beginning and at the end of an axis) along axis $j$.
\end{enumerate}

\noindent
For instance, the above examples shows a  $2\times 2$ kernel/filter $\bm{W}$  applied to a $3 \times 3$ input padded with a $0 \times 0$
border of zeros using $1 \times 1$ strides.

Note that strides constitute a form of \textbf{subsampling}. As an alternative to
being interpreted as a measure of how much the kernel/filter is translated, strides
can also be viewed as how much of the output is retained. For instance, moving
the kernel by hops of two is equivalent to moving the kernel by hops of one but
retaining only odd output elements.

% !split
\subsection*{Pooling}

In addition to discrete convolutions themselves, {\em pooling\/} operations
make up another important building block in CNNs. Pooling operations reduce
the size of feature maps by using some function to summarize subregions, such
as taking the average or the maximum value.

Pooling works by sliding a window across the input and feeding the content of
the window to a {\em pooling function}. In some sense, pooling works very much
like a discrete convolution, but replaces the linear combination described by
the kernel with some other function. Poolin
provides an example for average pooling, and 
does the same for max pooling.

The following properties affect the output size $o_j$ of a pooling layer
along axis $j$:

\begin{enumerate}
\item $i_j$: input size along axis $j$,

\item $k_j$: pooling window size along axis $j$,

\item $s_j$: stride (distance between two consecutive positions of the pooling window) along axis $j$.
\end{enumerate}

\noindent
The analysis of the relationship between convolutional layer properties is eased
by the fact that they don't interact across axes, i.e., the choice of kernel
size, stride and zero padding along axis $j$ only affects the output size of
axis $j$. Because of that, we will focus on the following simplified
setting:

\begin{enumerate}
\item 2-D discrete convolutions ($N = 2$),

\item square inputs ($i_1 = i_2 = i$),

\item square kernel size ($k_1 = k_2 = k$),

\item same strides along both axes ($s_1 = s_2 = s$),

\item same zero padding along both axes ($p_1 = p_2 = p$).
\end{enumerate}

\noindent
This facilitates the analysis and the visualization, but keep in mind that the
results outlined here also generalize to the N-D and non-square cases.

% !split
\subsection*{No zero padding, unit strides}

The simplest case to analyze is when the kernel just slides across every
position of the input (i.e., $s = 1$ and $p = 0$).

For any $i$ and $k$, and for $s = 1$ and $p = 0$,
\begin{equation*}
    o = (i - k) + 1.
\end{equation*}

% !split
\subsection*{Zero padding, unit strides}

To factor in zero padding (i.e., only restricting to $s = 1$), let's consider
its effect on the effective input size: padding with $p$ zeros changes the
effective input size from $i$ to $i + 2p$. In the general case, we can infer the following
relationship

For any $i$, $k$ and $p$, and for $s = 1$,
\begin{equation*}
    o = (i - k) + 2p + 1.
\end{equation*}

% !split
\subsection*{Half (same) padding}

Having the output size be the same as the input size (i.e., $o = i$) can be a
desirable property:

For any $i$ and for $k$ odd ($k = 2n + 1, \quad n \in \mathbb{N}$), $s = 1$ and
$p = \lfloor k / 2 \rfloor = n$,
\begin{equation*}
\begin{split}
    o &= i + 2 \lfloor k / 2 \rfloor - (k - 1) \\
      &= i + 2n - 2n \\
      &= i.
\end{split}
\end{equation*}

% !split
\subsection*{Full padding}

While convolving a kernel generally decreases the output size with
respect to the input size, sometimes the opposite is required. This can be
achieved with proper zero padding:

For any $i$ and $k$, and for $p = k - 1$ and $s = 1$,
\begin{equation*}
\begin{split}
    o &= i + 2(k - 1) - (k - 1) \\
      &= i + (k - 1).
\end{split}
\end{equation*}

This is sometimes referred to as full padding, because in this
setting every possible partial or complete superimposition of the kernel on the
input feature map is taken into account. 

% !split
\subsection*{Pooling arithmetic}

In a neural network, pooling layers provide invariance to small translations of
the input. The most common kind of pooling is \textbf{max pooling}, which
consists in splitting the input in (usually non-overlapping) patches and
outputting the maximum value of each patch. Other kinds of pooling exist, e.g.,
mean or average pooling, which all share the same idea of aggregating the input
locally by applying a non-linearity to the content of some patches.

Since pooling does not involve
zero padding, the relationship describing the general case is as follows:

For any $i$, $k$ and $s$,
\begin{equation*}
    o = \left\lfloor \frac{i - k}{s} \right\rfloor + 1.
\end{equation*}

% !split
\subsection*{Building convolutional neural networks in Tensorflow and Keras}

As discussed above, CNNs are neural networks built from the assumption that the inputs
to the network are 2D images. This is important because the number of features or pixels in images
grows very fast with the image size, and an enormous number of weights and biases are needed in order to build an accurate network.  

As before, we still have our input, a hidden layer and an output. What's novel about convolutional networks
are the \textbf{convolutional} and \textbf{pooling} layers stacked in pairs between the input and the hidden layer.
In addition, the data is no longer represented as a 2D feature matrix, instead each input is a number of 2D
matrices, typically 1 for each color dimension (Red, Green, Blue). 

% !split
\subsection*{Setting it up}

It means that to represent the entire
dataset of images, we require a 4D matrix or \textbf{tensor}. This tensor has the dimensions:  
\[  
(n_{inputs},\, n_{pixels, width},\, n_{pixels, height},\, depth) .
\]

% !split
\subsection*{The MNIST dataset again}

The MNIST dataset consists of grayscale images with a pixel size of
$28\times 28$, meaning we require $28 \times 28 = 724$ weights to each
neuron in the first hidden layer.

If we were to analyze images of size $128\times 128$ we would require
$128 \times 128 = 16384$ weights to each neuron. Even worse if we were
dealing with color images, as most images are, we have an image matrix
of size $128\times 128$ for each color dimension (Red, Green, Blue),
meaning 3 times the number of weights $= 49152$ are required for every
single neuron in the first hidden layer.


% !split
\subsection*{Strong correlations}

Images typically have strong local correlations, meaning that a small
part of the image varies little from its neighboring regions. If for
example we have an image of a blue car, we can roughly assume that a
small blue part of the image is surrounded by other blue regions.

Therefore, instead of connecting every single pixel to a neuron in the
first hidden layer, as we have previously done with deep neural
networks, we can instead connect each neuron to a small part of the
image (in all 3 RGB depth dimensions).  The size of each small area is
fixed, and known as a \href{{https://en.wikipedia.org/wiki/Receptive_field}}{receptive}.


% !split 
\subsection*{Layers of a CNN}
The layers of a convolutional neural network arrange neurons in 3D: width, height and depth.  
The input image is typically a square matrix of depth 3. 

A \textbf{convolution} is performed on the image which outputs
a 3D volume of neurons. The weights to the input are arranged in a number of 2D matrices, known as \textbf{filters}.

Each filter slides along the input image, taking the dot product
between each small part of the image and the filter, in all depth
dimensions. This is then passed through a non-linear function,
typically the \textbf{Rectified Linear (ReLu)} function, which serves as the
activation of the neurons in the first convolutional layer. This is
further passed through a \textbf{pooling layer}, which reduces the size of the
convolutional layer, e.g.~by taking the maximum or average across some
small regions, and this serves as input to the next convolutional
layer.

% !split
\subsection*{Systematic reduction}

By systematically reducing the size of the input volume, through
convolution and pooling, the network should create representations of
small parts of the input, and then from them assemble representations
of larger areas.  The final pooling layer is flattened to serve as
input to a hidden layer, such that each neuron in the final pooling
layer is connected to every single neuron in the hidden layer. This
then serves as input to the output layer, e.g.~a softmax output for
classification.


% !split
\subsection*{Prerequisites: Collect and pre-process data}









































\begin{minted}[fontsize=\fontsize{9pt}{9pt},linenos=false,mathescape,baselinestretch=1.0,fontfamily=tt,xleftmargin=7mm]{python}
# import necessary packages
import numpy as np
import matplotlib.pyplot as plt
from sklearn import datasets


# ensure the same random numbers appear every time
np.random.seed(0)

# display images in notebook
%matplotlib inline
plt.rcParams['figure.figsize'] = (12,12)


# download MNIST dataset
digits = datasets.load_digits()

# define inputs and labels
inputs = digits.images
labels = digits.target

# RGB images have a depth of 3
# our images are grayscale so they should have a depth of 1
inputs = inputs[:,:,:,np.newaxis]

print("inputs = (n_inputs, pixel_width, pixel_height, depth) = " + str(inputs.shape))
print("labels = (n_inputs) = " + str(labels.shape))


# choose some random images to display
n_inputs = len(inputs)
indices = np.arange(n_inputs)
random_indices = np.random.choice(indices, size=5)

for i, image in enumerate(digits.images[random_indices]):
    plt.subplot(1, 5, i+1)
    plt.axis('off')
    plt.imshow(image, cmap=plt.cm.gray_r, interpolation='nearest')
    plt.title("Label: %d" % digits.target[random_indices[i]])
plt.show()

\end{minted}


% !split
\subsection*{Importing Keras and Tensorflow}























\begin{minted}[fontsize=\fontsize{9pt}{9pt},linenos=false,mathescape,baselinestretch=1.0,fontfamily=tt,xleftmargin=7mm]{python}
from tensorflow.keras import datasets, layers, models
from tensorflow.keras.layers import Input
from tensorflow.keras.models import Sequential      #This allows appending layers to existing models
from tensorflow.keras.layers import Dense           #This allows defining the characteristics of a particular layer
from tensorflow.keras import optimizers             #This allows using whichever optimiser we want (sgd,adam,RMSprop)
from tensorflow.keras import regularizers           #This allows using whichever regularizer we want (l1,l2,l1_l2)
from tensorflow.keras.utils import to_categorical   #This allows using categorical cross entropy as the cost function
#from tensorflow.keras import Conv2D
#from tensorflow.keras import MaxPooling2D
#from tensorflow.keras import Flatten

from sklearn.model_selection import train_test_split

# representation of labels
labels = to_categorical(labels)

# split into train and test data
# one-liner from scikit-learn library
train_size = 0.8
test_size = 1 - train_size
X_train, X_test, Y_train, Y_test = train_test_split(inputs, labels, train_size=train_size,
                                                    test_size=test_size)

\end{minted}


% !split 
\subsection*{Running with Keras}




























\begin{minted}[fontsize=\fontsize{9pt}{9pt},linenos=false,mathescape,baselinestretch=1.0,fontfamily=tt,xleftmargin=7mm]{python}
def create_convolutional_neural_network_keras(input_shape, receptive_field,
                                              n_filters, n_neurons_connected, n_categories,
                                              eta, lmbd):
    model = Sequential()
    model.add(layers.Conv2D(n_filters, (receptive_field, receptive_field), input_shape=input_shape, padding='same',
              activation='relu', kernel_regularizer=regularizers.l2(lmbd)))
    model.add(layers.MaxPooling2D(pool_size=(2, 2)))
    model.add(layers.Flatten())
    model.add(layers.Dense(n_neurons_connected, activation='relu', kernel_regularizer=regularizers.l2(lmbd)))
    model.add(layers.Dense(n_categories, activation='softmax', kernel_regularizer=regularizers.l2(lmbd)))
    
    sgd = optimizers.SGD(lr=eta)
    model.compile(loss='categorical_crossentropy', optimizer=sgd, metrics=['accuracy'])
    
    return model

epochs = 100
batch_size = 100
input_shape = X_train.shape[1:4]
receptive_field = 3
n_filters = 10
n_neurons_connected = 50
n_categories = 10

eta_vals = np.logspace(-5, 1, 7)
lmbd_vals = np.logspace(-5, 1, 7)

\end{minted}


% !split
\subsection*{Final part}


















\begin{minted}[fontsize=\fontsize{9pt}{9pt},linenos=false,mathescape,baselinestretch=1.0,fontfamily=tt,xleftmargin=7mm]{python}
CNN_keras = np.zeros((len(eta_vals), len(lmbd_vals)), dtype=object)
        
for i, eta in enumerate(eta_vals):
    for j, lmbd in enumerate(lmbd_vals):
        CNN = create_convolutional_neural_network_keras(input_shape, receptive_field,
                                              n_filters, n_neurons_connected, n_categories,
                                              eta, lmbd)
        CNN.fit(X_train, Y_train, epochs=epochs, batch_size=batch_size, verbose=0)
        scores = CNN.evaluate(X_test, Y_test)
        
        CNN_keras[i][j] = CNN
        
        print("Learning rate = ", eta)
        print("Lambda = ", lmbd)
        print("Test accuracy: %.3f" % scores[1])
        print()

\end{minted}


% !split
\subsection*{Final visualization}
































\begin{minted}[fontsize=\fontsize{9pt}{9pt},linenos=false,mathescape,baselinestretch=1.0,fontfamily=tt,xleftmargin=7mm]{python}
# visual representation of grid search
# uses seaborn heatmap, could probably do this in matplotlib
import seaborn as sns

sns.set()

train_accuracy = np.zeros((len(eta_vals), len(lmbd_vals)))
test_accuracy = np.zeros((len(eta_vals), len(lmbd_vals)))

for i in range(len(eta_vals)):
    for j in range(len(lmbd_vals)):
        CNN = CNN_keras[i][j]

        train_accuracy[i][j] = CNN.evaluate(X_train, Y_train)[1]
        test_accuracy[i][j] = CNN.evaluate(X_test, Y_test)[1]

        
fig, ax = plt.subplots(figsize = (10, 10))
sns.heatmap(train_accuracy, annot=True, ax=ax, cmap="viridis")
ax.set_title("Training Accuracy")
ax.set_ylabel("$\eta$")
ax.set_xlabel("$\lambda$")
plt.show()

fig, ax = plt.subplots(figsize = (10, 10))
sns.heatmap(test_accuracy, annot=True, ax=ax, cmap="viridis")
ax.set_title("Test Accuracy")
ax.set_ylabel("$\eta$")
ax.set_xlabel("$\lambda$")
plt.show()

\end{minted}


% !split
\subsection*{The CIFAR01 data set}

The CIFAR10 dataset contains 60,000 color images in 10 classes, with
6,000 images in each class. The dataset is divided into 50,000
training images and 10,000 testing images. The classes are mutually
exclusive and there is no overlap between them.













\begin{minted}[fontsize=\fontsize{9pt}{9pt},linenos=false,mathescape,baselinestretch=1.0,fontfamily=tt,xleftmargin=7mm]{python}
import tensorflow as tf

from tensorflow.keras import datasets, layers, models
import matplotlib.pyplot as plt

# We import the data set
(train_images, train_labels), (test_images, test_labels) = datasets.cifar10.load_data()

# Normalize pixel values to be between 0 and 1 by dividing by 255. 
train_images, test_images = train_images / 255.0, test_images / 255.0


\end{minted}


% !split
\subsection*{Verifying the data set}

To verify that the dataset looks correct, let's plot the first 25 images from the training set and display the class name below each image.
















\begin{minted}[fontsize=\fontsize{9pt}{9pt},linenos=false,mathescape,baselinestretch=1.0,fontfamily=tt,xleftmargin=7mm]{python}
class_names = ['airplane', 'automobile', 'bird', 'cat', 'deer',
               'dog', 'frog', 'horse', 'ship', 'truck']
​
plt.figure(figsize=(10,10))
for i in range(25):
    plt.subplot(5,5,i+1)
    plt.xticks([])
    plt.yticks([])
    plt.grid(False)
    plt.imshow(train_images[i], cmap=plt.cm.binary)
    # The CIFAR labels happen to be arrays, 
    # which is why you need the extra index
    plt.xlabel(class_names[train_labels[i][0]])
plt.show()

\end{minted}


% !split
\subsection*{Set up  the model}

The 6 lines of code below define the convolutional base using a common pattern: a stack of Conv2D and MaxPooling2D layers.

As input, a CNN takes tensors of shape (image_height, image_width, color_channels), ignoring the batch size. If you are new to these dimensions, color_channels refers to (R,G,B). In this example, you will configure our CNN to process inputs of shape (32, 32, 3), which is the format of CIFAR images. You can do this by passing the argument input_shape to our first layer.












\begin{minted}[fontsize=\fontsize{9pt}{9pt},linenos=false,mathescape,baselinestretch=1.0,fontfamily=tt,xleftmargin=7mm]{python}
model = models.Sequential()
model.add(layers.Conv2D(32, (3, 3), activation='relu', input_shape=(32, 32, 3)))
model.add(layers.MaxPooling2D((2, 2)))
model.add(layers.Conv2D(64, (3, 3), activation='relu'))
model.add(layers.MaxPooling2D((2, 2)))
model.add(layers.Conv2D(64, (3, 3), activation='relu'))

# Let's display the architecture of our model so far.

model.summary()

\end{minted}


You can see that the output of every Conv2D and MaxPooling2D layer is a 3D tensor of shape (height, width, channels). The width and height dimensions tend to shrink as you go deeper in the network. The number of output channels for each Conv2D layer is controlled by the first argument (e.g., 32 or 64). Typically, as the width and height shrink, you can afford (computationally) to add more output channels in each Conv2D layer.

% !split
\subsection*{Add Dense layers on top}

To complete our model, you will feed the last output tensor from the
convolutional base (of shape (4, 4, 64)) into one or more Dense layers
to perform classification. Dense layers take vectors as input (which
are 1D), while the current output is a 3D tensor. First, you will
flatten (or unroll) the 3D output to 1D, then add one or more Dense
layers on top. CIFAR has 10 output classes, so you use a final Dense
layer with 10 outputs and a softmax activation.








\begin{minted}[fontsize=\fontsize{9pt}{9pt},linenos=false,mathescape,baselinestretch=1.0,fontfamily=tt,xleftmargin=7mm]{python}
model.add(layers.Flatten())
model.add(layers.Dense(64, activation='relu'))
model.add(layers.Dense(10))
Here's the complete architecture of our model.

model.summary()

\end{minted}

As you can see, our (4, 4, 64) outputs were flattened into vectors of shape (1024) before going through two Dense layers.

% !split
\subsection*{Compile and train the model}









\begin{minted}[fontsize=\fontsize{9pt}{9pt},linenos=false,mathescape,baselinestretch=1.0,fontfamily=tt,xleftmargin=7mm]{python}
model.compile(optimizer='adam',
              loss=tf.keras.losses.SparseCategoricalCrossentropy(from_logits=True),
              metrics=['accuracy'])
​
history = model.fit(train_images, train_labels, epochs=10, 
                    validation_data=(test_images, test_labels))


\end{minted}


% !split
\subsection*{Finally, evaluate the model}













\begin{minted}[fontsize=\fontsize{9pt}{9pt},linenos=false,mathescape,baselinestretch=1.0,fontfamily=tt,xleftmargin=7mm]{python}
plt.plot(history.history['accuracy'], label='accuracy')
plt.plot(history.history['val_accuracy'], label = 'val_accuracy')
plt.xlabel('Epoch')
plt.ylabel('Accuracy')
plt.ylim([0.5, 1])
plt.legend(loc='lower right')

test_loss, test_acc = model.evaluate(test_images,  test_labels, verbose=2)

print(test_acc)


\end{minted}



% ------------------- end of main content ---------------

\end{document}

