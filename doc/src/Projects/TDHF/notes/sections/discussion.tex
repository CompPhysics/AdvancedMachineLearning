Considering at a first stage the time-independent Hartree-Fock solver, we can see that our code provided us with a faithful reproduction of the results contained in the paper from Zangellini et al. Both the general and restricted representation implemented on our side revealed to be effective in this task: in fact the total energy values appearing in Table \ref{tab:E_x_comp_article} and the one-body density shown in Eq.\,\ref{fig:one_body_density_comp} perfectly match the estimated provided in the article. In addition, the two expected values for the position operator appear to be on the order of $10^{-10}$ in magnitude, thus reasonably compatible with the null value. Furthermore, as expected, we observe that the integral of the one-body densities gives us the number of particles. These result were not reported in the article, but still they are consistent with the symmetrical shape of the one body density and its expected normalization. These procedures acted also as a verification of the correct implementation of the various functions, giving us a high confidence in using the ground state coefficient matrix as a seed for the subsequently performed time evolution.\\

Further analysis of the convergence properties of the general and restricted Hartree-Fock solvers revealed some interesting aspects. Considering the results for the energy evaluated at every iteration reported in Figure \ref{fig:energy_at_every_step}, we can notice a first plateau reached after few steps by both the systems, followed by a split of the two curves. The apparatus described by the general representation explores a region of lower energy, converging to a more stable state after some iterations. This is proved also by the behaviour of the corresponding $\Delta$ parameter reported in Figure \ref{fig:delta_at_every_step}: here we can appreciate the fact that the smallest $\Delta$ values appear in correspondence of the second plateau in the energy diagram. A little grow around the $70$-th iteration occurs, this probably being a symptom of some instability of the algorithm. However, it is worth noticing that the convergence to a lower energy state follows as a consequence of the spin mixing allowed in the general representation: in this context, the system will converge to the energetically most favourable state, which cannot be explored by the other system, since the restricted implementation does not allow for spin mixing. 

In this second case, the adopted solver reaches better stability within fewer iterations, as clearly seen also in Figure \ref{fig:delta_at_every_step}. We also notice an inversion of the slope for $\Delta$ around the $40$-th iteration, corresponding then to an inversion in the energy curve as the number of steps increases. These unexpected behaviours can reasonably be symptoms of instabilities of the algorithm, since the variational principle that drove the derivation of the Hartree-Fock theory seems to be even no more satisfied. 

Concluding the discussion about the time-independent solver, we can say that the imposition of a lower tolerance in the general solver would have led the convergence to a lower energy state. However, we still opted to keep the tolerance to $\delta=10^{-6}$, in order to be able to make some comparisons with the content of the article by Zanghellini et al. \\


Switching then to the time domain, the overlap curve presented in Figure \ref{fig:overlap_comp_paper} almost perfectly matches the one presented in the article: this being a further confirmation of the goodness of the implemented code also when time enters into play. Plotting the time evolution of the overlap for a long time allowed us to appreciate the almost perfect symmetry of the $\xi(t)$ curve, which results to be specular with respect to the vertical line passing in $t\omega/2\pi=4$. In particular, after a whole period of oscillation of $2\pi/\Omega$ inside the harmonic potential, we can see that the system returns very close to its original state. The frequency with which this occurs is, in our case, the same as the frequency of the harmonic oscillator in which the electrons are confined. However, a prolonged plot showed that after a few periods the symmetry of the curve was lost: we didn't analyze in depth the reasons behind this, which could however consist in the finite precision of the integrator or more simply in a natural deviation of the system from the symmetrical behaviour. \\ 

Other tests performed with different values of $\omega$ revealed that this symmetric behaviour occurs for small $t$ independently from the chosen $\omega$, under the condition that $\omega \gtrsim 8\Omega$. This can possibly be explained by analyzing the effect brought in by the laser source: the time-dependent potential introduced in Eq.\,\ref{eq:hamiltonian_t_dip} indeed acts on the system by modifying at each time instant the position and the depth of the minimum in the harmonic oscillator potential. The higher is the frequency of the laser source, the harder is for the system to adapt rapidly to the changes induced in the single particle potential before an oscillation in the laser term is actually completed. The system will then spend much time close to the initial state, as shown also in Figure \ref{fig:overlap_comp_omega}: here one appreciates the fact that the overlap curve assumes on average higher values in correspondence of large $\omega$.  On the contrary, when the values of $\omega$ and $\Omega$ are comparable, the effects introduced by the laser source are much more evident since the system gets close to a resonant behaviour. In this last case, the symmetry of the curve is broken. For a better visualization of these concepts, please refer to the animated pictures reported \href{https://github.com/Matteo294/FYS4411/tree/main/Project2/code}{here}.

The considerations about the symmetry described up to now are very evident also in Figure \ref{fig:dipole_comp}. The oscillations in the curves for $\omega=8\Omega$ and $\omega=32\Omega$ follow a sinusoidal behaviour with a frequency $\Omega$, while the little wiggles are attributable to the component provided by the laser source with a frequency $\omega$. As this last parameter increases in value, the curve appears to be more an more similar to a pure sinusoidal wave at frequency $\Omega$, while evident early breaks in the symmetry induced by the resonant behaviour appears for small $\omega$ values. \\


Exploiting again the time dependent solver, we performed another evolution of the system with the laser kept in operation for a limited period and then switching it off for the rest of the simulation. The curve obtained for $\overline{x}(t)$ reported in Figure \ref{fig:laser_on_off} shows the already described behaviour for $t<T$, then becoming much more similar to a pure sinusoidal wave when the laser is switched off. The same behaviour also occurs for other combinations of $\Omega$ and $\omega$ and appears to be evident in the frequency spectra presented in Figure \ref{fig:fourier_spectra_x}. Here the main peaks should appear in correspondence of the transition energy values between the various excited states of the considered time-independent system, but, as we see, just one sharp line appears. We could suppose that the absence of other lines is due to the form of Eq.\,\ref{eq:transition_energies}: here each complex exponential oscillating at frequency $\omega_{ij}$ is multiplied by the corresponding $\bracketOP{\Psi_i}{\Psi_j}{\hat{x}}$. It may occur that this expected value is null for certain combinations of $i$ and $j$, annihilating also the corresponding contributions in the spectrum for the frequency of interest in this context. \\

Finally, the curve for $\xi_T(t)$ presented in Figure \ref{fig:overlap_T} is not so informative about the transition frequencies, thus for a better comprehension we must recur to the actual spectra presented in Figure \ref{fig:fourier_spectra_xi}. Here many more frequencies appear in the spectrum, the majority of them probably caused by some noise in the overlap curves induced by the discrete time integration. However, among them it is possible to distinguish some higher peaks: further analysis may be performed to better analyze the origin of these peaks, especially for what concerns their relation with the frequency $\Omega$. Indeed, they appear to be approximately equally spaced of a factor $\Omega$, suggesting some strong link with the typical spectrum of an harmonic oscillator. Moreover, the intensity of the peaks goes diminishing for increasing values of $\omega_{fft}$, symptom of the fact that the state in which we leave the system after the laser has been switched off is given by a superposition of different states, with major contribution given by those possessing lower energy. 

% In the first part of the evolution, both the particles are forced to follow the minimum of the harmonic oscillator potential, which moves at every time instant under the effect brought in by the laser source. Later in time, the harmonic hole returns to be steadily centered in $x=0$ and the two electrons are then free to oscillate into the hole itself, giving rise to the sinusoidal part of the curve for $t>T$. In this last part of the evolution, the two electrons' behaviour becomes comparable with a classical one, with the mentioned oscillations that we are used to see associated to a classical particle inserted in a harmonic potential.

% This similarity with the classical behaviour appears even more evident when we come then to the analysis of the frequency components generating the signal for $t>T$. We can notice from Figure \ref{fig:fourier_spectra} that the main peak is given in each case by the frequency associated to the harmonic oscillator potential and no correlations with the value of $\omega$ appear. 