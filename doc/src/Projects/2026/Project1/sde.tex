\documentclass[11pt,a4paper]{article}
\usepackage{amsmath,amssymb,physics}
\usepackage{graphicx}
\usepackage{hyperref}
\usepackage{geometry}
\usepackage{enumitem}
\geometry{margin=2.5cm}

\title{\textbf{Methodological Roadmap for Machine Learning-Based Inference of Lévy Jump-Diffusion Processes}}
\author{}
\date{}

\begin{document}
\maketitle

\section*{1. Scientific Objective}

The objective of this project is to develop and compare machine learning methodologies for estimating the parameters and structure of Lévy jump-diffusion processes underlying financial asset prices.

We consider stochastic processes of the form
\begin{equation}
dS_t = \mu S_t dt + \sigma S_t dW_t + S_{t^-} dJ_t,
\end{equation}
where
\begin{itemize}
    \item $\mu$ is the drift,
    \item $\sigma^2$ is diffusion variance,
    \item $W_t$ is Brownian motion,
    \item $J_t$ is a compound Poisson jump process with intensity $\lambda$ and jump distribution $\nu(dx)$.
\end{itemize}

The parameter vector is
\[
\theta = (\mu, \sigma^2, \lambda, \nu).
\]

The task is to infer $\theta$ (or approximate the underlying distribution) from discrete-time observations of $S_t$.

\section*{2. Mathematical Framework}

\subsection*{2.1 Kolmogorov Forward Equation}

The probability density $p(x,t)$ satisfies a partial integro-differential equation (PIDE):

\begin{equation}
\partial_t p = -\mu \partial_x p + \frac{\sigma^2}{2}\partial_x^2 p 
+ \lambda \int_{\mathbb{R}} \left[ p(x-y,t) - p(x,t) \right] \nu(dy).
\end{equation}

This provides a deterministic constraint that can be embedded in physics-informed neural networks.

\subsection*{2.2 Inverse Problem}

Given data $\mathcal{D} = \{S_{t_i}\}_{i=1}^N$, estimate:

\[
\theta^* = \arg\max_\theta p(\mathcal{D} \mid \theta),
\]

or approximate the posterior distribution:

\[
p(\theta \mid \mathcal{D}).
\]

\section*{3. Methodological Architecture}

The project is structured in five stages.

\section*{Stage I: Synthetic Data Generation}

\begin{itemize}
    \item Implement simulation of Lévy processes.
    \item Generate datasets across parameter regimes.
    \item Study identifiability and sensitivity.
    \item Validate statistical estimators (MLE, method of moments).
\end{itemize}

Deliverable: Baseline statistical inference benchmark.

\section*{Stage II: Supervised Neural Network Estimation}

\subsection*{2A. Direct Parameter Regression}

Train networks:
\[
\text{Time Series} \rightarrow (\mu, \sigma^2, \lambda, \text{jump parameters})
\]

Architectures:
\begin{itemize}
    \item MLP (baseline)
    \item 1D CNN
    \item LSTM / GRU
    \item Transformer encoder
\end{itemize}

Loss function:
\[
\mathcal{L} = \sum_i \norm{\hat{\theta}_i - \theta_i}^2.
\]

Evaluation:
\begin{itemize}
    \item Parameter estimation error
    \item Sensitivity to sampling frequency
    \item Robustness under model misspecification
\end{itemize}

\section*{Stage III: Physics-Informed Neural Networks (PINNs)}

Instead of regressing parameters directly, approximate $p(x,t)$ by a neural network $p_\phi(x,t)$.

Loss function:

\begin{equation}
\mathcal{L} = \mathcal{L}_{data} + \alpha \mathcal{L}_{PIDE},
\end{equation}

where

\begin{equation}
\mathcal{L}_{PIDE} = \norm{
\partial_t p_\phi 
+ \mu \partial_x p_\phi
- \frac{\sigma^2}{2}\partial_x^2 p_\phi
- \lambda \int (p_\phi(x-y)-p_\phi(x)) \nu(dy)
}^2.
\end{equation}

Goals:
\begin{itemize}
    \item Enforce physical consistency
    \item Improve generalization
    \item Study stability of integro-differential operator learning
\end{itemize}

\section*{Stage IV: Bayesian Machine Learning}

\subsection*{4A. Bayesian Neural Networks}

Place priors on weights and infer posterior:

\[
p(\theta \mid \mathcal{D})
\]

via:
\begin{itemize}
    \item Variational inference
    \item Monte Carlo dropout
    \item Hamiltonian Monte Carlo (if feasible)
\end{itemize}

\subsection*{4B. Gaussian Process Hybrid Models}

Model:
\[
X_t = \text{GP diffusion} + \text{Sparse jump process}
\]

Study:
\begin{itemize}
    \item Jump detection
    \item Volatility decomposition
    \item Uncertainty quantification
\end{itemize}

Deliverable: Credible intervals for parameters.

\section*{Stage V: Neural Stochastic Differential Equations}

Generalize to neural SDE framework:

\begin{equation}
dX_t = f_\theta(X_t,t) dt + g_\theta(X_t,t) dW_t + dJ_t.
\end{equation}

Objectives:
\begin{itemize}
    \item Learn drift and diffusion functions non-parametrically.
    \item Compare structured Lévy assumption vs neural SDE flexibility.
    \item Study overfitting vs interpretability trade-offs.
\end{itemize}

\section*{4. Comparative Evaluation Framework}

All methods will be benchmarked against:

\begin{itemize}
    \item Maximum likelihood estimation
    \item Expectation–maximization methods
    \item Classical jump-diffusion calibration
\end{itemize}

Metrics:

\begin{itemize}
    \item Parameter RMSE
    \item Log-likelihood on unseen data
    \item Predictive density calibration
    \item Computational cost
    \item Uncertainty quantification quality
\end{itemize}

\section*{5. Application to Real Financial Data}

\begin{itemize}
    \item Calibrate on synthetic data.
    \item Apply to historical equity and FX data.
    \item Compare parameter estimates to standard econometric methods.
    \item Study stability across market regimes.
\end{itemize}

\section*{6. Risk Analysis}

\begin{itemize}
    \item Identifiability issues for $\lambda$ at low frequency.
    \item Jump distribution non-uniqueness.
    \item Overfitting of neural SDE models.
    \item Numerical instability in PINN integro-differential terms.
\end{itemize}

Mitigation:
\begin{itemize}
    \item Regularization.
    \item Bayesian priors.
    \item Cross-validation.
    \item Theoretical error bounds.
\end{itemize}

\section*{7. Expected Scientific Contributions}

\begin{itemize}
    \item Systematic comparison of ML paradigms for Lévy inference.
    \item Demonstration of PINNs for integro-differential stochastic systems.
    \item Uncertainty-aware neural parameter estimation.
    \item Insights into interpretability vs flexibility trade-offs.
\end{itemize}

\section*{8. Publication Strategy}

Potential venues:

\begin{itemize}
    \item Quantitative Finance
    \item Journal of Computational Finance
    \item SIAM Journal on Financial Mathematics
    \item Machine Learning in Finance
    \item NeurIPS / ICML workshops (if neural SDE focus)
\end{itemize}

\section*{9. Timeline (12–18 Months)}

\begin{itemize}
    \item Months 1–3: Simulation + classical benchmarks.
    \item Months 4–6: Supervised neural models.
    \item Months 7–9: PINN development.
    \item Months 10–12: Bayesian extensions.
    \item Months 13–15: Neural SDE models.
    \item Months 16–18: Real data application + publication.
\end{itemize}

\section*{Conclusion}

This roadmap provides a structured, progressively sophisticated
approach to applying machine learning to Lévy jump-diffusion
inference. The project balances statistical rigor, physical
constraints, uncertainty quantification, and modern deep learning
architectures. It offers both methodological innovation and practical
financial relevance.

\end{document}

