%%
%% Automatically generated file from DocOnce source
%% (https://github.com/doconce/doconce/)
%% doconce format latex Project1.do.txt --print_latex_style=trac --latex_admon=paragraph
%%
% #ifdef PTEX2TEX_EXPLANATION
%%
%% The file follows the ptex2tex extended LaTeX format, see
%% ptex2tex: https://code.google.com/p/ptex2tex/
%%
%% Run
%%      ptex2tex myfile
%% or
%%      doconce ptex2tex myfile
%%
%% to turn myfile.p.tex into an ordinary LaTeX file myfile.tex.
%% (The ptex2tex program: https://code.google.com/p/ptex2tex)
%% Many preprocess options can be added to ptex2tex or doconce ptex2tex
%%
%%      ptex2tex -DMINTED myfile
%%      doconce ptex2tex myfile envir=minted
%%
%% ptex2tex will typeset code environments according to a global or local
%% .ptex2tex.cfg configure file. doconce ptex2tex will typeset code
%% according to options on the command line (just type doconce ptex2tex to
%% see examples). If doconce ptex2tex has envir=minted, it enables the
%% minted style without needing -DMINTED.
% #endif

% #define PREAMBLE

% #ifdef PREAMBLE
%-------------------- begin preamble ----------------------

\documentclass[%
oneside,                 % oneside: electronic viewing, twoside: printing
final,                   % draft: marks overfull hboxes, figures with paths
10pt]{article}

\listfiles               %  print all files needed to compile this document

\usepackage{relsize,makeidx,color,setspace,amsmath,amsfonts,amssymb}
\usepackage[table]{xcolor}
\usepackage{bm,ltablex,microtype}

\usepackage[pdftex]{graphicx}

\usepackage{ptex2tex}
% #ifdef MINTED
\usepackage{minted}
\usemintedstyle{default}
% #endif

\usepackage[T1]{fontenc}
%\usepackage[latin1]{inputenc}
\usepackage{ucs}
\usepackage[utf8x]{inputenc}

\usepackage{lmodern}         % Latin Modern fonts derived from Computer Modern

% Hyperlinks in PDF:
\definecolor{linkcolor}{rgb}{0,0,0.4}
\usepackage{hyperref}
\hypersetup{
    breaklinks=true,
    colorlinks=true,
    linkcolor=linkcolor,
    urlcolor=linkcolor,
    citecolor=black,
    filecolor=black,
    %filecolor=blue,
    pdfmenubar=true,
    pdftoolbar=true,
    bookmarksdepth=3   % Uncomment (and tweak) for PDF bookmarks with more levels than the TOC
    }
%\hyperbaseurl{}   % hyperlinks are relative to this root

\setcounter{tocdepth}{2}  % levels in table of contents

% prevent orhpans and widows
\clubpenalty = 10000
\widowpenalty = 10000

% --- end of standard preamble for documents ---


% insert custom LaTeX commands...

\raggedbottom
\makeindex
\usepackage[totoc]{idxlayout}   % for index in the toc
\usepackage[nottoc]{tocbibind}  % for references/bibliography in the toc

%-------------------- end preamble ----------------------

\begin{document}

% matching end for #ifdef PREAMBLE
% #endif

\newcommand{\exercisesection}[1]{\subsection*{#1}}


% ------------------- main content ----------------------



% ----------------- title -------------------------

\thispagestyle{empty}

\begin{center}
{\LARGE\bf
\begin{spacing}{1.25}
Project 1
\end{spacing}
}
\end{center}

% ----------------- author(s) -------------------------

\begin{center}
{\bf \href{{https://www.uio.no/studier/emner/matnat/fys/FYS5429/index-eng.html}}{FYS5429/9429}, Advanced machine learning and data analysis for the physical sciences, University of Oslo, Norway${}^{}$} \\ [0mm]
\end{center}

\begin{center}
% List of all institutions:
\end{center}
    
% ----------------- end author(s) -------------------------

% --- begin date ---
\begin{center}
Spring semester 2025, deadline March 21
\end{center}
% --- end date ---

\vspace{1cm}


\section{Possible paths for project 1}

We discuss here several paths as well as data sets for the first project (or as parts of a larger project)
Tentative deadline March 22. We would also like to propose that people who have formed groups present eventual projects on January 30. 

\subsection{Paths for the projects}

\begin{enumerate}
\item The computational path: Here we propose a path where you develop your own code for a convolutional or eventually recurrent neural network and apply this to data selects of your own selection. The code should be object oriented and flexible allowing for eventual extensions by including different Loss/Cost functions and other functionalities. Feel free to select data sets from those suggested below here. This code can also be extended upon by adding for example autoencoders. You can compare your own codes with implementations using TensorFlow(Keras)/PyTorch or other libraries. An alternative is to develop a code for an RNN inclusing long-term-short-memory (LSTM). This is often the starting point for large-language models and so-called transformers. Large language models are hovever not discussed during the lectures.

\item The differential equation path: Here we propose a set of differential equations (ordinary and/or partial) to be solved first using neural networks (using either your own code or TensorFlow/Pytorch or similar libraries). Thereafter we plan to extend the set of methods for solving these equations to recurrent neural networks and autoencoders (AE). This project can also be extended into including \href{{https://github.com/maziarraissi/PINNs}}{Physics informed machine learning}. Here we can discuss neural networks that are trained to solve supervised learning tasks while respecting any given law of physics described by general nonlinear partial differential equations. 

\item The application path: Here you can use the most relevant method(s) (say convolutional neural networks for images) and apply this(these) to data sets relevant for your own research.

\item The Gaussian processes/Bayesian statistics path: \href{{https://jenfb.github.io/bkmr/overview.html}}{Kernel regression (Gaussian processes) and Bayesian statistics} are  popular tools in the machine learning literature. The main idea behind these approaches is to flexibly model the relationship between a large number of variables and a particular outcome (dependent variable). This can form a second part of a project where for example standard Kernel regression methods are used on a specific data set. Alternatively, participants can opt to work on a large project relevant for their own research using gaussian processes and/or Bayesian machine Learning. 

\item Other possibilities.
\end{enumerate}

\noindent
\subsection{Defining the data sets to analyze yourself}

You can propose own data sets that relate to your research interests or just use existing data sets from say
\begin{enumerate}
\item \href{{https://www.kaggle.com/datasets}}{Kaggle} 

\item The \href{{https://archive.ics.uci.edu/ml/index.php}}{University of California at Irvine (UCI) with its  machine learning repository}.

\item For the differential equation problems, you can generate your own datasets, as described below.

\item If possible, you should link the data sets with existing research and analyses thereof. Scientific articles which have used Machine Learning algorithms to analyze the data are highly welcome. Perhaps you can improve previous analyses and even publish a new article? 

\item A critical assessment of the methods with ditto perspectives and recommendations is also something you need to include. For those of you familiar with report writing, the layouts discussed in for example \href{{https://github.com/CompPhysics/MachineLearning/blob/master/doc/Projects/EvaluationGrading/EvaluationForm.md}}{FYS-STK4155}.
\end{enumerate}

\noindent
\subsection{Solving  differential equations with neural networks}

Here we describe the possible differential equations we can study
first with neural networks and thereafter with recurrent neural
networks and/or Autoenconders.

The differential equations are given by the so-called \href{{https://encyclopediaofmath.org/index.php?title=Lorenz_attractor}}{Lorenz attractor model}, and read

\[
\frac{dx}{dt}=\sigma\left(y-x\right),
\]
where $\sigma =10$ is a constant
\[
\frac{dy}{dt}= x\left(\rho-z\right)-y,
\]
with $\rho=28$ and
\[
\frac{dz}{dt}=xy-\beta z
\]

with $\beta=8/3$ as our final constant.

The following function is a
simple function which sets up the solution using the ordinary
differential library which follows \textbf{NumPy}. Here we have fixed the
time sted $\Delta t=0.01$ and the final time $t_f=8$.

The program sets $100$ random initial values and produces inputs and outputs for a neural network calculations.
The inputs are given by the values of the array $\bm{x}$ (which contains $x,y,z$ as functions of time) for the time step $\bm{x}_t$.
The other array defined by $\bm{x}_{t+1}$ contains the outputs (or targets) which we want the neural network to reproduce.


























































\bpycod
# Common imports
import numpy as np
from scipy.integrate import odeint
import matplotlib.pyplot as plt
import os

# Where to save the figures and data files
PROJECT_ROOT_DIR = "Results"
FIGURE_ID = "Results/FigureFiles"
DATA_ID = "DataFiles/"

if not os.path.exists(PROJECT_ROOT_DIR):
    os.mkdir(PROJECT_ROOT_DIR)

if not os.path.exists(FIGURE_ID):
    os.makedirs(FIGURE_ID)

if not os.path.exists(DATA_ID):
    os.makedirs(DATA_ID)

def image_path(fig_id):
    return os.path.join(FIGURE_ID, fig_id)

def data_path(dat_id):
    return os.path.join(DATA_ID, dat_id)

def save_fig(fig_id):
    plt.savefig(image_path(fig_id) + ".png", format='png')


# Selection of parameter values and setting array for time
dt =0.01; tfinal = 8
t = np.arange(0,tfinal+dt, dt)
beta =8.0/3.0; rho = 28.0; sigma = 10.0

# define the inputs and outputs for the neural networks
nninput = np.zeros((100*len(t)-1,3))
nnoutput = np.zeros((100*len(t)-1,3))
# Define the equations to integrate
def lorenz_derivative(xyz, t0, sigma=sigma,beta=beta,rho=rho):
    x, y, z = xyz
    return [sigma*(x-y), x*(rho-z)-y, x*y-beta*z]

# generate 100 random initial values
x0 = -15.0+30.0*np.random.random((100,3))

# Use odeint functionality by sending in derivative function
# Feel free to change the choice of integrator
x_t = np.asarray([odeint(lorenz_derivative, x0_j, t) 
                  for x0_j in x0])

# define the inputs and outputs for the neural networks
for j in range(100):
    nninput[j*(len(t)-1):(j+1)*(len(t)-1),:] = x_t[j,:-1,:]
    nnoutput[j*(len(t)-1):(j+1)*(len(t)-1),:] = x_t[j,1:,:]


\epycod


The input and output variables are those we will start trying our
network with. Your first taks is to set up a neural code (either using
your own code or TensorFlow/PyTorch or similar libraries)) and use the
above data to a prediction for the time evolution of Lorenz system for
various values of the randomly chosen initial values.  Study the
dependence of the fit as function of the architecture of the network
(number of nodes, hidden layers and types of activation functions) and
various regularization schemes and optimization methods like standard
gradient descent with momentum, stochastic gradient descent with
batches and with and without momentum and various schedulers for the
learning rate.

Feel free to change the above differential equations. As an example,
consider the following harmonic oscillator equations solved with the
Runge-Kutta to fourth order method. This is a one-dimensional problem
and it produces a position $x_t$ and velocity $v_t$. You could now try
to fit both the velocities and positions using much of the same recipe
as for Lorenz attractor.  You will find it convenient to analyze one
set of initial conditions first. The code is included here.

This code is an example code that solves Newton's equations of motion
with a given force and produces an output which in turn can be used to
train a neural network































































































\bpycod
# Common imports
import numpy as np
import pandas as pd
from math import *
import matplotlib.pyplot as plt
import os

# Where to save the figures and data files
PROJECT_ROOT_DIR = "Results"
FIGURE_ID = "Results/FigureFiles"
DATA_ID = "DataFiles/"

if not os.path.exists(PROJECT_ROOT_DIR):
    os.mkdir(PROJECT_ROOT_DIR)

if not os.path.exists(FIGURE_ID):
    os.makedirs(FIGURE_ID)

if not os.path.exists(DATA_ID):
    os.makedirs(DATA_ID)

def image_path(fig_id):
    return os.path.join(FIGURE_ID, fig_id)

def data_path(dat_id):
    return os.path.join(DATA_ID, dat_id)

def save_fig(fig_id):
    plt.savefig(image_path(fig_id) + ".png", format='png')


def SpringForce(v,x,t):
#   note here that we have divided by mass and we return the acceleration
    return  -2*gamma*v-x+Ftilde*cos(t*Omegatilde)


def RK4(v,x,t,n,Force):
    for i in range(n-1):
# Setting up k1
        k1x = DeltaT*v[i]
        k1v = DeltaT*Force(v[i],x[i],t[i])
# Setting up k2
        vv = v[i]+k1v*0.5
        xx = x[i]+k1x*0.5
        k2x = DeltaT*vv
        k2v = DeltaT*Force(vv,xx,t[i]+DeltaT*0.5)
# Setting up k3
        vv = v[i]+k2v*0.5
        xx = x[i]+k2x*0.5
        k3x = DeltaT*vv
        k3v = DeltaT*Force(vv,xx,t[i]+DeltaT*0.5)
# Setting up k4
        vv = v[i]+k3v
        xx = x[i]+k3x
        k4x = DeltaT*vv
        k4v = DeltaT*Force(vv,xx,t[i]+DeltaT)
# Final result
        x[i+1] = x[i]+(k1x+2*k2x+2*k3x+k4x)/6.
        v[i+1] = v[i]+(k1v+2*k2v+2*k3v+k4v)/6.
        t[i+1] = t[i] + DeltaT


# Main part begins here

DeltaT = 0.001
#set up arrays 
tfinal = 20 # in dimensionless time
n = ceil(tfinal/DeltaT)
# set up arrays for t, v, and x
t = np.zeros(n)
v = np.zeros(n)
x = np.zeros(n)
# Initial conditions (can change to more than one dim)
x0 =  1.0 
v0 = 0.0
x[0] = x0
v[0] = v0
gamma = 0.2
Omegatilde = 0.5
Ftilde = 1.0
# Start integrating using Euler's method
# Note that we define the force function as a SpringForce
RK4(v,x,t,n,SpringForce)

# Plot position as function of time    
fig, ax = plt.subplots()
ax.set_ylabel('x[m]')
ax.set_xlabel('t[s]')
ax.plot(t, x)
fig.tight_layout()
save_fig("ForcedBlockRK4")
plt.show()


\epycod


The next step is to include recurrent neural networks. These will be discussed in connection with coming lectures.

\subsection{Introduction to numerical projects}

Here follows a brief recipe and recommendation on how to write a report for each
project.

\begin{itemize}
  \item Give a short description of the nature of the problem and the eventual  numerical methods you have used.

  \item Describe the algorithm you have used and/or developed. Here you may find it convenient to use pseudocoding. In many cases you can describe the algorithm in the program itself.

  \item Include the source code of your program. Comment your program properly.

  \item If possible, try to find analytic solutions, or known limits in order to test your program when developing the code.

  \item Include your results either in figure form or in a table. Remember to        label your results. All tables and figures should have relevant captions        and labels on the axes.

  \item Try to evaluate the reliabilty and numerical stability/precision of your results. If possible, include a qualitative and/or quantitative discussion of the numerical stability, eventual loss of precision etc.

  \item Try to give an interpretation of you results in your answers to  the problems.

  \item Critique: if possible include your comments and reflections about the  exercise, whether you felt you learnt something, ideas for improvements and  other thoughts you've made when solving the exercise. We wish to keep this course at the interactive level and your comments can help us improve it.

  \item Try to establish a practice where you log your work at the  computerlab. You may find such a logbook very handy at later stages in your work, especially when you don't properly remember  what a previous test version  of your program did. Here you could also record  the time spent on solving the exercise, various algorithms you may have tested or other topics which you feel worthy of mentioning.
\end{itemize}

\noindent
\subsection{Format for electronic delivery of report and programs}

The preferred format for the report is a PDF file. You can also use DOC or postscript formats or as an ipython notebook file.  As programming language we prefer that you choose between C/C++, Fortran2008 or Python. The following prescription should be followed when preparing the report:

\begin{itemize}
  \item Send us an email in order  to hand in your projects with a link to your GitHub/Gitlab repository.

  \item In your GitHub/GitLab or similar repository, please include a folder which contains selected results. These can be in the form of output from your code for a selected set of runs and input parameters.
\end{itemize}

\noindent
Finally, 
we encourage you to collaborate. Optimal working groups consist of 
2-3 students. You can then hand in a common report. 


% ------------------- end of main content ---------------

% #ifdef PREAMBLE
\end{document}
% #endif

